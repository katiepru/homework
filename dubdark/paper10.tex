\documentclass[12pt]{article}
\usepackage[margin=1.2in]{geometry}
\usepackage{indentfirst}

\begin{document}

\noindent Kaitlin Poskaitis\\
Assignment 11\\
November 22, 2013
\begin{center}
    \section*{\bf Star Lives and Death}
\end{center}

Back in the nineteenth century, many scientists were studying the Sun and it's
properties.  They were trying to determine things such as its age, its expected
lifespan, and other properties that explain its behavior.  One of these
scientists was Lord Kelvin, who determined that the Sun must have some sort of
power source.  He deduced this by assuming that the Sun could not have any
properties that matter on earth could not have.  He therefore thought of the Sun
as a large ball of iron or rock for comparison, and as a result determined that
it must have a source of fuel in order to give off the amount of energy that it
does.  Some time later, scientists discovered what this fuel source is:
hydrogen. Hydrogen fusion occurs in the core of the Sun and other stars of a
similar size. The fusion of hydrogen and that of most other elements smaller
than iron is an exothermic process, which explains how the Sun can radiate some
much energy.  However, only about half of this energy is radiated, while the
other half keeps the inside of the Sun hot enough to avoid shrinking from
cooling.

The Sun, like any other star, has a limited amount of fuel and will therefore
run out of it someday.  Scientists believe that the Sun will continue to shine
for about 5 billion years, as it has already shone for 4.5 billion years and has
used up a bit under half of its fuel.  When the Sun begins to die, the helium in
its core from hydrogen fusion will begin to burn.  This will cause the Sun to
become a red giant while doing hydrogen and helium fusion to make carbon.  After
this, the outer layers of gas will strip off by forming a planetary nebula, and
then the star will become a white dwarf, which is very small but very hot and
dense. It would contain most of the mass of the Sun while being about the size
of the Earth.  Stars with similar masses as that of the Sun will have a similar
fate.

Stars that are much heavier than the Sun have very different fates.  Their
higher mass means that they can continue to do fusion further along the periodic
table up to and including iron.  Once an iron core forms, it is supported by
electron degeneracy pressure.  Electrons continue to gain energy until they are
able to convert all protons to neutrons and cause a core collapse, which ends in
a supernova.  If this star was not too large, under a few solar masses, this
process will result in a neutron star.  A neutron star is another stellar
remnant that is much denser than a white dwarf and is held up by neutron
degeneracy pressure.  It may be as massive as the Sun but only about 10
kilometers in diameter.  If the star was larger, this means the resulting
neutron stars would not be able to support itself through neutron degeneracy
pressure, and a stellar mass black hole would result.

One example of observational evidence that we have for white dwarves is Sirius
B.  This is a white dwarf located next to Sirius A in our night sky, with Sirius
A being the brightest star we can see.  However, when the night sky is observed
in the X-ray, Sirius B appears much brighter and even drowns out most of Sirius
A.  This is because white dwarves radiate at much shorter wavelengths.  Stellar
mass black holes, however, cannot be observed so directly.  One black hole
candidate that scientists have found is the companion of Cygnus X-1.  Its
companion in the binary system is very small but massive, too massive for even a
neutron star.  Therefore, this companion must be a black hole.

The first stars in the universe were more massive than those that we see today.
Scientists estimate that they were between 300-1000 solar masses in size.  They
were able to get so massive and luminous because only hydrogen and helium were
present at that time.  It is believed that these early stars caused a process
called reionization.  This was a point in the universe's history where a lot of
visible matter reverted back to a plasma state.  The large amounts of radiation
from these early stars could have provided enough energy for this to happen.
Scientists believe that only 400 photons needed to be radiated for each baryonic
particle, which early stars may have been able to do.  This process is called
reionization because baryonic matter was already in this plasma state after the
big bang, and inflation caused the universe to cool enough for electrons to join
atoms in a process known as recombination.

By studying Lyman Alpha absorption lines, scientists have found that much of the
intergalactic medium was plasma starting from around redshift 6.5.  The process
of reionization was not an instant or uniform one, though.  It is believed that
radiation produced from large early stars created bubbles of plasma amongst
neutral gas.  In areas more densely populated with stars, such as galaxies and
galaxy clusters, these bubble combined to form supermassive ionizing bubbles.
As more bubbles formed and merged, more of the universe became reionized until
the entire universe was reionized.

\end{document}
