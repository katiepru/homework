\documentclass[12pt]{article}
\usepackage[margin=1.1in]{geometry}
\usepackage{indentfirst}

\begin{document}

\noindent Kaitlin Poskaitis\\
Assignment 9\\
November 1, 2013
\begin{center}
    \section*{\bf Inhabitants of the Local Universe}
\end{center}

%Intro
The Milky Way, our home galaxy, is one of many different types of galaxies that
we have observed.  It has several distinct parts found in many other spiral
galaxies, and by studying these parts we can learn about the galaxy's formation.
We study our galaxy at many different wavelengths to get information about it
that is not determinable by just studying its visible radiation.  Our galaxy is
one of many in the Local Group, which we study to learn about other types of
galaxies.  These galaxies and other nearby galaxies can be classified in many
different ways due to their large range in properties.

%Parts of Milky Way
The Milky Way can be broken down into several distinct components that can give
scientists information about its properties and even formation.  First, the
most luminous parts of the galaxy are engulfed in what is known as the stellar
halo (Lecture 10/29).  This is a mostly spherical region around 100 kpc in size,
which is much
larger than the other visible components.  The stars in this region are
generally very sparse.  Furthermore, they are very old, metal-poor stars.
Globular clusters can also be found here (Lecture 10/29).

Another part of our galaxy is the central bulge. This is the very luminous
center of the galaxy that forms a visual bulge in the center of the disk
portion.  It contains about one quarter of the total mass of the galaxy and
consists primarily of old, metal-rich stars (Lecture 10/29).  There is also
a bar of stars that
goes through the central bulge and is more flat.

The central bulge forms the center of the thin
and thick rotating disks, which are other separate components of the galaxy
(Lecture 10/29).
There are different stars in each of these two disks: the thin disk has younger,
more metal-rich stars while the thick disk has older, more metal-poor stars.
These disks make up most of the mass of the galaxy, with the thin disk being
much larger and more massive than the thick disk.  Finally, the spiral arms,
which form the outer portions of the luminous disks,
are composed of young stars, gas, and dust (Lecture 10/29).

%Formation
The nature of these structures described can give scientists some insight into
how the galaxy was formed.  For example, there are distinct streaks of stars in
the stellar halo that suggest that the Milky Way collided with many dwarf
galaxies which would have gotten torn apart by the core's gravity (Finkbeiner,
2012).  As a result,
we believe that the stellar halo was formed from disrupted dwarf galaxies and
the disk was formed from other dwarf galaxies being torn apart at the center.
These mergers of many dwarf galaxies is a controversial area in astronomy right
now because there is evidence seemingly for it and against it.  For example, the
streaks found in the stellar halo described above support this idea.  However,
the fact that the disk is intact is in tension with this idea because even
slight disturbances would cause the disk to become malformed greatly
(Finkbeiner, 2012).

%Star gas star cycle
Another important point to note about galaxy formation and evolution is the
formation and death of stars.  Galaxies would not be able to exist as we know
them for very long with forming new stars or acquiring new stars from elsewhere.
The cycle of star birth, death, and rebirth is called the star-gas-star cycle
(Lecture 10/29).
Essentially this outlines the process by which gas forms stars initially.  Then,
stellar feedback from the newly created stars will disrupt the dense gas cloud
around, preventing it from forming new stars for some time.  These stars
eventually return a lot of the gas to the pool of gas around them as they
evolve, which then form new stars (Lecture 10/29).

%Different wavelengths
When scientists study the Milky Way, they make sure to do so at the many
different
wavelengths of light that can be detected (Lecture 10/29).  This is because
observing it at
different wavelengths can tell us different information about the galaxy that we
cannot easily obtain just by studying visible light.  Gas and dust clouds will
distort our view when looking at visible wavelengths, but other wavelengths can
show the different types of gas, molecular and atomic, as well as the different
types of stars that are present in the galaxy.  This is information that would
be nearly impossible to determine by just studying visible light because of the
gas distortion (Lecture 10/29).

%Local group
Our galaxy is part of a group of nearby galaxies called the Local Group.  This
group consists of several different types of galaxies, each of which has unique
properties.  We are one of two big spiral galaxies, with Andromeda being the
other.  Andromeda is essentially a scaled up version of the Milky Way, as it is
more luminous, bigger, and more massive (Lecture 10/29).  It also has a
relatively larger center
bulge.  It's stellar halo also has large streams like those found in the Milky
Way, only 100 kpc in length, suggesting that it accreted a very large galaxy at
some point in the past.  There is also one medium sized spiral galaxy called
M33, which is essentially a scaled down version of the Milky Way.  It is more
gas-rich and has a smaller bulge.  Also, it does not have a black hole at the
center, but instead has a nuclear star cluster (Lecture 10/29).

The Local Group does not contain any giant elliptical galaxies, but does contain
an odd galaxy that is classified as a mini giant elliptical (Lecture 10/31).
It's name is M32,
and it is the most luminous satellite of M33.  It does not have any cool gas,
young stars, or globular clusters, has a very high central luminosity, contains
mostly metal-rich stars, and has a supermassive black hole.  These properties
are evidence that it is the stripped nucleus of a much larger galaxy (Lecture
10/31).

There are also 10 irregular and dwarf irregular galaxies (Lecture 10/29).  These
galaxies are
much less luminous than the other types of galaxies in the Local Group.  The
irregular galaxies also have much more random star motion as opposed to directed
motion.  They are typically gas-rich, have globular clusters, and do not have a
stellar halo.  The dwarf irregulars consists of mostly young stars and a lot of
gas.  They are very diffuse and also show the more random star motions (Lecture
10/29).

Lastly, there are at least 20-30 dwarf galaxies in the Local Group (Lecture
10/29).  These
galaxies have very low luminosity and surface brightness.  They contain mostly
old, metal-poor stars as well as very little gas.  They tend to be oval with
very little rotation, and they are dark matter dominated.  Otherwise, they are
very similar to globular clusters in nature.  Some of these galaxies, labelled
as transition galaxies, are ultra faint and therefore extremely difficult to
detect.  As a result, there may be many more of these dwarf galaxies that we
cannot detect with current technology (Lecture 10/29).

%Further galaxies
Beyond the Local Group, there are many different types of galaxies that can be
found in the nearby universe (Lecture 10/31).  These galaxies can be classified
by many of their
different properties.  For example, the Hubble way of classifying galaxies by
their shape is often used.  Galaxies can also be classified by properties such
as color, ellipticity, ratio of the central bulge to the disk, size, luminosity,
and so on (Lecture 10/31).

%Conclusion
In all, our galaxy has several distinct components and properties that allow us
to predict its history, although our current predictions of dwarf galaxy mergers
has some problems. We can study these galaxies' star formation and other
properties at different wavelengths.  We can also study the Local Group and the
nearby universe, which has shown us many different types of galaxies with many
different classifications.

\end{document}
