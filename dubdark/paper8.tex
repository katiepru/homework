\documentclass[12pt]{article}
\usepackage[margin=1.1in]{geometry}
\usepackage{indentfirst}

\begin{document}

\noindent Kaitlin Poskaitis\\
Assignment 9\\
November 1, 2013
\begin{center}
    \section*{\bf Inhabitants of the Local Universe}
\end{center}

%Parts of Milky Way

The Milky Way can be broken down into several distinct components that can give
scientists information about its properties and even formation.  First, the
most luminous parts of the galaxy are engulfed in what is known as the stellar
halo.  This is a mostly spherical region around 100 kpc in size, which is much
larger than the other visible components.  The stars in this region are
generally very sparse.  Furthermore, they are very old, metal-poor stars.
Globular clusters can also be found here.

Another part of our galaxy is the central bulge. This is the very luminous
center of the galaxy that forms a visual bulge in the center of the disk
portion.  It contains about one quarter of the total mass of the galaxy and
consists primarily of old, metal-rich stars.  There is also a bar of stars that
goes through the central bulge and is more flat.

The central bulge forms the center of the thin
and thick rotating disks, which are other separate components of the galaxy.
There are different stars in each of these two disks: the thin disk has younger,
more metal-rich stars while the thick disk has older, more metal-poor stars.
These disks make up most of the mass of the galaxy, with the thin disk being
much larger and more massive than the thick disk.  Finally, the spiral arms,
which form the outer portions of the luminous disks,
are composed of young stars, gas, and dust.

%Formation
The nature of these structures described can give scientists some insight into
how the galaxy was formed.  For example, there are distinct streaks of stars in
the stellar halo that suggest that the Milky Way collided with many dwarf
galaxies which would have gotten torn apart by the core's gravity.  As a result,
we believe that the stellar halo was formed from disrupted dwarf galaxies and
the disk was formed from other dwarf galaxies being torn apart at the center.
These mergers of many dwarf galaxies is a controversial area in astronomy right
now because there is evidence seemingly for it and against it.  For example, the
streaks found in the stellar halo described above support this idea.  However,
the fact that the disk is intact is in tension with this idea because even
slight disturbances would cause the disk to become malformed greatly.

%Star gas star cycle
Another important point to note about galaxy formation and evolution is the
formation and death of stars.  Galaxies would not be able to exist as we know
them for very long with forming new stars or acquiring new stars from elsewhere.
The cycle of star birth, death, and rebirth is called the star-gas-star cycle.
Essentially this outlines the process by which gas forms stars initially.  Then,
stellar feedback from the newly created stars will disrupt the dense gas cloud
around, preventing it from forming new stars for some time.  These stars
eventually return a lot of the gas to the pool of gas around them as they
evolve, which then form new stars.

%Different wavelengths
When scientists study the Milky Way, they make sure to do so at the many
different
wavelengths of light that can be detected.  This is because observing it at
different wavelengths can tell us different information about the galaxy that we
cannot easily obtain just by studying visible light.  Gas and dust clouds will
distort our view when looking at visible wavelengths, but other wavelengths can
show the different types of gas, molecular and atomic, as well as the different
types of stars that are present in the galaxy.  This is information that would
be nearly impossible to determine by just studying visible light because of the
gas distortion.

\end{document}
