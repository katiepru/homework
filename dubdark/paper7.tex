\documentclass[12pt]{article}
\usepackage[margin=1.1in]{geometry}
\usepackage{indentfirst}

\begin{document}

\noindent Kaitlin Poskaitis\\
Assignment 8\\
October 25, 2013
\begin{center}
    \section*{\bf Stars and the discovery of Galaxies}
\end{center}


%Star formation
Scientists believe that most stars are born through the compression of gas
clouds.  Essentially, large clouds of gas in a galaxy can become cooler as they
lose energy, which causes them to compact.  This raises their density, and
eventually a gas cloud can become so dense that it collapses on itself.  When
this occurs, every section of the cloud pulls every other section closer
together.  This rapidly raises the density, which in turn causes the temperature
to rise.  Eventually, the temperature will rise to about 10 million degrees,
which is the minimum temperature for nuclear fusion to occur.  This nuclear
fusion is what causes the star to radiate light as it gives off an extreme
amount of energy.

%Size
Because of this process by which stars form, stars can only have masses within a
specific range.  Specifically, the minimum mass is about one tenth of a solar
mass, while the maximum mass is about 100 solar masses.  The minimum mass can be
explained easily by the process outlined above.  Essentially, if the mass is any
less then the cloud cannot reach a high enough temperature for nuclear fusion to
occur, so it remains dark.  The maximum mass, however, is caused by a different
phenomenon entirely.  If a star's mass is too large, it will become luminous
enough that the radiation pushes any external gas away from it.  This also
causes the gas within the star to be pushed out of it, which eventually leads to
new star formation away from that massive star.

%Age
Stars have many other properties along with their mass that can give scientists
great insight into their properties.  For example, scientists can use a star's
spectrum to determine its age.  This is done by determining the ratio of lithium
to the rest if the gas in the star.  Each star starts with some amount of
lithium left over from the Big Bang, and the nuclear fusion that occurs within
the stars destroys it over time.  In other words, stars with a higher percentage
of lithium are typically younger.  The mass and temperature of a star also can
determine how its lifetime will go.  More massive stars tend to be hotter, and
these stars have much shorter lifetimes.  Conversely, less massive stars tend to
be cooler and have much longer lifetimes.

%Sun vs massive star
Besides the length of a star's life, a star's mass also determines what stages
it will go through in its lifetime.  For example, our sun, an average sized star,
was born and is currently undergoing nuclear fusion.  Eventually, it will become
what is called a red giant, where its outer layers of gas will expand a
hundredfold and get swept away, leaving only a core 

\end{document}
