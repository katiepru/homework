\documentclass[12pt]{article}
\usepackage[margin=1.1in]{geometry}
\usepackage{indentfirst}

\begin{document}

\noindent Kaitlin Poskaitis\\
Assignment 8\\
October 25, 2013
\begin{center}
    \section*{\bf Stars and the discovery of Galaxies}
\end{center}

%Intro
Our understanding of some of the larger structures in the universe has grown
dramatically throughout the past few centuries.  We now have a good understanding
of stars in that we believe we know how the form and how their properties are
related.  We also have a decent understanding of their life cycles from watching
stars form and die around us.  We also have a decent understanding of nebulae,
or large gas clouds, that began to take root in the 18th century with a debate
between two scientists named Curtis and Shapely.

%Star formation
Scientists believe that most stars are born through the compression of gas
clouds (Tyson and Goldsmith 2005).  Essentially, large clouds of gas in a galaxy
can become cooler as they
lose energy, which causes them to compact.  This raises their density, and
eventually a gas cloud can become so dense that it collapses on itself.  When
this occurs, every section of the cloud pulls every other section closer
together.  This rapidly raises the density, which in turn causes the temperature
to rise.  Eventually, the temperature will rise to about 10 million degrees,
which is the minimum temperature for nuclear fusion to occur.  This nuclear
fusion is what causes the star to radiate light as it gives off an extreme
amount of energy (Tyson and Goldsmith 2005).

%Size
Because of this process by which stars form, stars can only have masses within a
specific range.  Specifically, the minimum mass is about one tenth of a solar
mass, while the maximum mass is about 100 solar masses (Tyson and Goldsmith
2005).  The minimum mass can be
explained easily by the process outlined above.  Essentially, if the mass is any
less then the cloud cannot reach a high enough temperature for nuclear fusion to
occur, so it remains dark.  The maximum mass, however, is caused by a different
phenomenon entirely.  If a star's mass is too large, it will become luminous
enough that the radiation pushes any external gas away from it.  This also
causes the gas within the star to be pushed out of it, which eventually leads to
new star formation away from that massive star (Tyson and Goldsmith 2005).

%Age
Stars have many other properties along with their mass that can give scientists
great insight into their properties.  For example, scientists can use a star's
spectrum to determine its age (Tyson and Goldsmith 2005).  This is done by
determining the ratio of lithium
to the rest if the gas in the star.  Each star starts with some amount of
lithium left over from the Big Bang, and the nuclear fusion that occurs within
the stars destroys it over time.  In other words, stars with a higher percentage
of lithium are typically younger.  The mass and temperature of a star also can
determine how its lifetime will go.  More massive stars tend to be hotter, and
these stars have much shorter lifetimes.  Conversely, less massive stars tend to
be cooler and have much longer lifetimes (Tyson and Goldsmith 2005).

%Sun vs massive star
Besides the length of a star's life, a star's mass also determines what stages
it will go through in its lifetime (Tyson and Goldsmith 2005).  For example, our
sun, an average sized star,
was born and is currently undergoing nuclear fusion.  Eventually, it will become
what is called a red giant, where its outer layers of gas will expand a
hundredfold and get swept away, leaving only a core of used fuel that will
remain dark.  On the other hand, a more massive star, say around 20 times as
massive as our sun, would have a very different life cycle.  Instead of becoming
a red giant and dissipating, the star would die in the form of a supernova.
This occurs when too much energy is released and blows the entire star apart,
dispersing the elements formed in its core throughout the galaxy (Tyson and
Goldsmith 2005).

%History of galaxies
Stars are not the only things that can be seen in the night sky.  Another class
of objects we can see is called nebulae.  Some objects that we now know fall
into this category are different types of gas clouds (Lecture 10/24).  There are
bright gas
clouds that glow which can be one of several subtypes.  Planetary nebulae are
round in shape and have a star in the center.  Reflection/emission nebulae
reflect or emit light from nearby stars, making them appear to glow.  Spiral or
elliptical nebulae refer to their associated types of galaxies (Lecture 10/24).

%History
Our current understanding of nebulae came from a rich history of misinterpreted
data (Lecture 10/24).  It all started with a man named Thomas Wright, who created
an idea of
island universes, but in a more religious sense than physical.  From his work, a
scientist named Kant came up with the idea that the Milky Way is disk-shaped and
then determined that we might not be the only galaxy out there.  However, his
work was not acknowledged until long after the fact because he did not present
his work to the king.  Instead, another scientist named Johann Heinrich Lambert
came up with essentially the same idea and was recognized for it (Lecture 10/24).

%more
From this idea, two fields of thought emerged regarding nebulae (Lecture 10/24).
There was the
theory of island universes, which was led by Curtis, which stated that the Sun
was in the middle of the Milky Way, the Milky Way is relatively small, and that
there are island universes (galaxies) outside of the Milky Way.  The other
school of thought was called the nebular hypothesis, led by Shapely, stated that
the Milky Way was the entire
universe, that the sun was not in the center, and that the observed spiral
nebulae were late forming solar systems.  The Island Universe theory did not
have any observational evidence for it initially, while the nebular hypothesis
believed they had observational evidence that turned out to be incorrectly
interpreted. One example of this is that they believed they could observe a
nebula rotating, which was not possible.  This debated was finally resolved when
Edwin Hubble found that Andromeda was $10^6$ light years away, which was too far
for the nebular hypothesis to hold (Lecture 10/24).

%Conclusion
In all, we believe we have a pretty good understanding of larger celestial
bodies such as stars and galaxies.  We know that stars are formed from gas cloud
compression, and that their mass plays a strong role in their lifetime.  We can
use certain properties of stars, such as spectra, to determine other properties
of theirs, such as age.  Some stars terminate by becoming red giants and
dissipating, while more massive stars die through supernovae that disperse heavy
elements throughout the universe.  Our understanding of galaxies and other
nebulae came from a great debate in history over whether the Milky Way was the
entire universe or not.  We found that to be untrue and thus our modern
understanding of nebulae began to form.

\end{document}
