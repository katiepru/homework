\documentclass[14pt]{article}
\usepackage[margin=1.2in]{geometry}
\usepackage{indentfirst}

\begin{document}

\noindent Kaitlin Poskaitis\\
September 19, 2013
\begin{center} \section*{\bf Astronomer’s Toolkit: Light, Telescopes, and Relativity}
\end{center}

\vspace{5mm}
% Intro
Astronomers have a very strong and varied arsenal of tools at their disposal to
study the cosmos.  Perhaps one of their strongest and most useful tools for this
is the telescope, which allows astronomers to study the light that celestial
bodies emits.  One may be able to imagine that telescopes are very useful tools
for a variety of reasons, but they are not the only tool astronomers can utilize
to help them understand celestial phenomena.  There is another set of tools
astronomers use that does not include physical instruments, rather, previous
theories.  Perhaps one of the most influential theories ever created in the
realm of physics is Einstein's theory of special and general relativity.  By
using both physical tools such as telescopes, along with intellectual tools such
as Einstein's theories, astronomers can better understand the universe.

\vspace{5mm}
% Telescope usefulness
Telescope work by allowing us to study the light that a body emits, which can
give us a plethora of information about that body.  We can measure emission
lines and absorption lines in order to determine several things, such as its
velocity, rotation speed, and temperature (Lecture 9/17.  We can measure an
object's velocity
by using the notion of Doppler shift.  If a body is coming towards us, the light
it emits will be blue-shifted, while if an object is moving away from us, the
light it emits will be red-shifted.  This is because if a body is moving away
from us, the light that we see has a lower frequency than normal, making the
light appear more red, and vice versa.  Measuring rotation speed is similar,
where scientists measure the Doppler shift of different parts along the body's
radius, and subtract the average velocity to create a rotation function.
Lastly, temperature can be measured by the wavelength of light a body emits.
Hotter objects tend to emit light with lower wavelengths, and cooler objects
tend to emit light with longer wavelengths (Lecture 9/17).

\vspace{5mm}
It is evident that telescopes can be used to gather great amounts of
information.  However, the power of a telescope depends on two things: its light
collecting area and its angular resolution, which is the
smallest area over which we can tell two objects are distinct (Lecture 9/17).
The light collecting area is determined by the telescope's diameter, and it
needs to be larger for longer wavelengths of light, such as microwaves.  The
angular resolution is also a function of diameter, and is very important because
a telescope with poor angular resolution could fail to distinguish between
bodies that are close to each other.  Even though telescopes benefit from a
higher diameter in both light collecting area and angular resolution, sometimes
building telescopes of that size is impractical, so instead a cluster of them
can be built near each other and their collective data can be used as if they
were one big telescope.  This makes building "large telescopes" much easier
because the array can be built over time, which makes it much more cost
effective (Lecture 9/17).

\vspace{5mm}
% Space telescopes
It is evident that telescopes are very powerful tools for studying our universe;
however, their results can vary greatly based on the location that they are
used.  Telescopes can be used on Earth or in space, and their are certain
advantages and disadvantages to each option.  If a telescope is to be used on
Earth, it must be used in a dry, dark, and high location (Lecture 9/17).  This
minimizes the effects of light pollution and atmospheric distortion.  Even so,
the atmosphere greatly limits the power of telescopes on Earth, as it not only
distorts light but also prevents certain wavelengths of light from ever reaching
the ground.  Even so, on-Earth telescopes are still much more common as they are
less expensive and much easier to maintain and upgrade.  Telescopes in space
have the advantage of being able to view celestial bodies more clearly, but lack
the ease of maintenance which often deters people from building them (Lecture
9/17).

\vspace{5mm}
There have been several famous space telescopes, perhaps the most well-known
being the Hubble Space Telescope.  It is a 2.4m telescope that was launched by
NASA in 1990 (Lecture, 9/17).  It specialized in optical, infrared, and ultraviolet
observations.  Chandra was another important space telescope, launched by NASA
in 1999.  It specialized in X-ray imaging and spectroscopy.  Lastly,  the
Wilkinson Microwave Anisotropy Probe was launched by NASA in 2011 and
specialized in studying the cosmic microwave background (Lecture, 9/17).

\vspace{5mm}
% Special relativity
It is obvious that telescopes are very useful tools for observing the cosmos;
however, it is the theories that have been developed that allow astronomers to
understand these observations.  One of the most important one of these theories
is Einstein's theory of special relativity, which describes some effects of two
main axioms: that the fundamentals of physics are the same for all observers,
and that the speed of light is constant for all observers, regardless of their
reference frame (velocity) (Lecture 9/19).  This has a few interesting effects,
the first being that time passes slower when you are moving faster, which is a
direct result of the speed of light being constant.  Another effect is that the
length of an object at higher velocities is shortened.  Lastly, an object at
higher velocities' mass is also greater.  It was from this increase in mass that
he was able to derive his most famous equation, $E=mc^2$, which is still
commonly used by astronomers today (Lecture 9/19).

\vspace{5mm}
% General relativity
Einstein's theory of special relativity has had many impacts on the field of
astronomy, it had one severe limitation: it only applied for objects at constant
velocities.  As a result, it does not apply for anything under the influence of
gravity, which causes acceleration.  In order to account for this, Einstein
created another theory for these conditions, called his theory of General
Relativity.  This theory is based on Einstein's equivalence principle, which
states that gravity is the same as any other acceleration, so something moving
between reference frames through gravity would do the same thing through
standard acceleration (Coles 2001).  In his general theory, Einstein embodied
the time relativity described in his special theory, but gravity further affects
things such as time dilation and length contraction.  More importantly though,
space is also affected in this theory.  In fact, he determined that gravity
could cause space to curve.  He also speculated that black holes could exist in
the center of galaxies (Coles 2001).

\vspace{5mm}
%Conclusion
In all, astronomers have a powerful and varied toolkit for studying the cosmos.
Technically advanced telescopes allow them to observe the light that celestial
bodies emit, which can then in turn allow them to conclude information such as
the body's velocity, rotational velocity, and temperature.  These telescopes can
be used on Earth or in space, and collect various types of data because of that.
On top of being able to use physical objects to examine the cosmos, they also
have at their disposal Einstein's theories of relativity, which allow
astronomers to analyze the data collected from telescopes and other means in a
meaningful way.  They can use both his theory of special relativity, which is
very insightful but only applies in situations where objects are moving at a
constant velocity, as well as his theory of general relativity, which can be
used in situations with acceleration.  Between the physical tools and Einstein's
theories, astronomers have a very powerful toolkit for studying our universe.
\end{document}
