\documentclass[12pt]{article}
\usepackage[margin=1.0in]{geometry}
\usepackage{indentfirst}

\begin{document}

\noindent Kaitlin Poskaitis\\
September 19, 2013
\begin{center}
\section*{\bf Astronomer’s Toolkit: Light, Telescopes, and Relativity}
\end{center}

% Intro
Astronomers have a very strong and varied arsenal of tools at their disposal to
study the cosmos.  Perhaps one of their strongest and most useful tools for this
is the telescope, which allows astronomers to study the light that celestial
bodies emits.  One may be able to imagine that telescopes are very useful tools
for a variety of reasons, but they are not the only tool astronomers can utilize
to help them understand celestial phenomena.  There is another set of tools
astronomers use that does not include physical instruments, rather, previous
theories.  Perhaps one of the most influential theories ever created in the
realm of physics is Einstein's theory of special and general relativity.  By
using both physical tools such as telescopes, along with intellectual tools such
as Einstein's theories, astronomers can better understand the universe.

% Telescope usefulness
Telescope work by allowing us to study the light that a body emits, which can
give us a plethora of information about that body.  We can measure emission
lines and absorption lines in order to determine several things, such as its
velocity, rotation speed, and temperature.  We can measure an object's velocity
by using the notion of Doppler shift.  If a body is coming towards us, the light
it emits will be blue-shifted, while if an object is moving away from us, the
light it emits will be red-shifted.  This is because if a body is moving away
from us, the light that we see has a lower frequency than normal, making the
light appear more red, and vice versa.  Measuring rotation speed is similar,
where scientists measure the Doppler shift of different parts along the body's
radius, and subtract the average velocity to create a rotation function.
Lastly, temperature can be measured by the wavelength of light a body emits.
Hotter objects tend to emit light with lower wavelengths, and cooler objects
tend to emit light with longer wavelengths.  In order to measure these
properties, the power of a telescope depends on two thing: its light collecting
area, which is its diameter, and its angular resolution, which is the smallest
area over which we can tell two objects are distinct, which is also based on
diameter.

% Space telescopes
It is evident that telescopes are very powerful tools for studying our universe;
however, their results can vary greatly based on the location that they are
used.  Telescopes can be used on Earth or in space, and their are certain
advantages and disadvantages to each option.  If a telescope is to be used on
Earth, it must be used in a dry, dark, and high location (Lecture 9/17).  This
minimizes the effects of light pollution and atmospheric distortion.  Even so,
the atmosphere greatly limits the power of telescopes on Earth, as it not only
distorts light but also prevents certain wavelengths of light from ever reaching
the ground.  Even so, on-Earth telescopes are still much more common as they are
less expensive and much easier to maintain and upgrade.  Telescopes in space
have the advantage of being able to view celestial bodies more clearly, but lack
the ease of maintenance which often deters people from building them.

There have been several famous space telescopes, perhaps the most well-known
being the Hubble Space Telescope.  It is a 2.4m telescope that was launched by
NASA in 1990 (Lecture, 9/17).  It specialized in optical, infrared, and ultraviolet
observations.  Chandra was another important space telescope, launched by NASA
in 1999.  It specialized in X-ray imaging and spectroscopy.  Lastly,  the
Wilkinson Microwave Anisotropy Probe was launched by NASA in 2011 and
specialized in studying the cosmic microwave background (Lecture, 9/17).

\end{document}
