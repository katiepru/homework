\documentclass[12pt]{article}
\usepackage[margin=1.1in]{geometry}
\usepackage{indentfirst}

\begin{document}

\noindent Kaitlin Poskaitis\\
Assignment 10\\
November 8, 2013
\begin{center}
    \section*{\bf Galaxy Properties and Galaxy Evolution}
\end{center}

If we were to look at all of the galaxies in the nearby universe ($z$<1), then we
would see a set of galaxies that can be classified in several ways.  It would be
possible to use Hubble's method of classifying galaxies by their morphology
fairly easily, especially because these are the galaxies he worked with to
create his classification system.  In general, there are three major types of
galaxy morphology: elliptical, spiral, and irregular.  There are specific
subcategories within each of these classifications as well, such as the
obviousness of a bar in a spiral or how tightly wound the arms are in a spiral.
Furthermore, these types of galaxies can be found in a variety of sizes, such as
giant, medium, and dwarf.

Hubble's classification method for galaxies works well within the nearby
universe, but it does not describe the only properties of galaxies that can be
used to classify them.  For example, galaxies can also be classified by other
properties such as their luminosity, the type of stars they contain, their
color, or their mass.  These are just some of many different properties which
galaxies could be classified by.

If you were to classify galaxies in the nearby universe by these properties, you
would notice some striking correlations between these properties.  For example,
you would notice that spiral galaxies tend to have much more young stars than
elliptical galaxies do.  As a result, spiral galaxies tend to be much bluer in
color, and also tend to be less massive than ellipticals.  Another important
correlation is that is notable is that the stars in spiral galaxies have higher
angular momentum than those in elliptical galaxies.  As a result, the stars in
spiral galaxies move in an orbital way, whereas the stars in elliptical galaxies
tend to follow random movement paths.

These were all examples of how intrinsic properties of galaxies can be
correlated, but a galaxy's properties are also highly dependent on its
environment.  A galaxy's environment can be described as what other large scale
structures are around it.  A common example would be if a galaxy is part of a
cluster, a gravitationally bound set of galaxies.  If a galaxy was part of a
cluster, this would mean that it was in a denser environment.  An interesting
correlation between the density of a galaxy's environment and its stars is that
galaxies in denser environments tend to have mostly older stars.  This is
because the gas found in and around these galaxies is hotter, which prevents
star formation as that requires cold gas.

These correlations between galaxy properties can be explained by several
theories.  One theory that remains strong today is the theory that states that
elliptical galaxies are formed by the merging of spiral galaxies.  This at first
seems like a pretty straightforward idea, but it has several important
consequences.  First, this could easily explain why elliptical galaxies are more
massive, as they would be the result of several spiral galaxies comparable in
size merging together.  This also can explain why stars in elliptical galaxies
have less angular momentum, as the merging process would have disrupted their
usual paths from the increased effects of gravity.  Lastly, it also explains why
the stars are older because the merging process takes billions of years, which
would imply that the stars have been around much longer.

This model fits reasonably well with our observations of older galaxies ($z$>1),
with some downfalls. This model would predict that if we looked back in time, we
would see less elliptical galaxies, the galaxies would be less massive, and they
would be bluer.  If we looked far enough back, we would not see any ellipticals
at all.  We do observe that the galaxies are bluer and less massive; however,
the galaxies do not look like the spirals we are used to observing.  This may be
explained by these galaxies being spiral precursors and have not fully developed
some of the features of our familiar spirals, such as distinct spiral arms.

\end{document}
