\documentclass[12pt]{article}
\usepackage[margin=1.0in]{geometry}
\usepackage{indentfirst}

\begin{document}

\noindent Kaitlin Poskaitis\\
Assignment 5\\
October 9, 2013
\begin{center}
    \section*{\bf The Very Early Universe}
\end{center}

%Intro
The origins of our universe have mystified people for thousands of years.  Since
the general acceptance of the big bang theory in modern society, we have begun
to understand some of the earliest moments of our universe.  However, even the
big bang theory cannot explain everything, and some phenomena could not be
explained by it alone.  The theory of inflation was proposed in order to solve
some of these problems, and was able to give us great insight into our
universe's earliest moments.

%Problems with big bang - ADD SOURCES
Although the big bang theory seemed to be the most accurate depiction of the
beginning of the universe, it gave rise to a few problems that could not easily
be explained.  The first of these problems is the problem of flatness.
Scientists have strong evidence for our universe having an $\Omega$ value very
close to 1 (Lecture 10/8).  However, if the early universe had even a tiny
amount of curvature,
that curvature would be amplified greatly by the expansion of the universe. It
seems extremely unlikely that the early universe had no curvature at all, so to
see the universe currently currently having almost no curvature is equally
unlikely.  The second problem that the big bang theory poses is that the
temperature of the observable universe is extremely close to being equivalent
everywhere.  Specifically, the temperature of the sky is the same to a part in
$1.0E^{-5}$, even on scales that should not have been causally connected in the
early universe.  The third issue with the big bang theory is that Grand Unified
Theories
predict that strange phenomena, such as monopoles, should exist.  The Grand
Unified Theories (GUT) refers to when there is enough energy to have the weak
force, the strong force, and the electromagnetic force all converge into one
force, which we believe occurred during the first fractions of a second after the
big bang (Lecture 10/8).

%Solutions!!
While these seem like relatively difficult problems to explain, the theory of
cosmic inflation provides an explanation for them all.  The theory of cosmic
inflation explains how hyperinflation of spacetime occurred during the first few
fractions of a second after the big bang (Lecture 10/8).  This provides a
solution to the
flatness problem because any spacetime is flat when you look at small enough
scales, which is what we believe is occurring.  Inflation also provides a
solution to the problems associated with the temperature distribution of the
night sky by saying that those parts which we thought would not be causally
connected in the early universe were in fact causally connected before
inflation.  Lastly, the problem of monopoles and other strange phenomena can be
solved by inflation because the GUT would have gotten too diluted for the
probabilities of these strange things existing to be significant (Lecture 10/8).

%Seeds and evidence

%Eternal inflation and multiverses
The theory of cosmic inflation is fairly well accepted amongst astronomers, but
it also has led to some less well defined but equally interesting theories,
specifically those of eternal inflation and the multiverse (Lecture 10/8).
Essentially, the
theory of eternal inflation suggests that there is a continuous inflation that
takes place in a quantum realm.  We cannot define time in this realm because the
scale is so small that things do not interact long enough for a concept of time
to develop.  This theory of eternal inflation also suggests that we are only a
small bubble in a larger complex of many universes, called the mulitverse.  This
is because quantum fluctuations may create bubbles in spacetime that expand out
into their own universes, and the universe that we live in happens to be a
stable one.  These other universes may have completely different laws of physics
as well, which suggests that our universe is extremely special in that it can
support life as we know it (Lecture 10/8).

%Very early universe
The theory of inflation has allowed scientists to be able to construct a
timeline of the very early universe, just a few minutes after the big bang.
Starting from just after the Planck time, which is the earliest time we could
possibly measure, the universe was extremely hot, on the order of $10^{34}$K
(Lecture 10/10).
Because of this extreme temperature, only the most fundamental particles would
be able to exist, as anything larger would be broken down into its components.
Some of these particles include quarks, leptons, and photons. There also was a
nearly equal amount of antimatter, although the amount of normal matter slightly
outnumbered it, which we know because normal matter exists today.  The early
universe was also much denser than it is today, such that photons could not
travel very far before bumping into something (Lecture 10/10).

%More
During this time of the very early universe is when inflation was occurring,
causing the universe to expand and therefore cool at astonishing rates (Lecture
10/10).  When
the temperature dropped by about $14$ orders of magnitude, the symmetry of
supergravity, or the unification of the four forces, broke, making the four
forces become independent.  This is when the Higgs particle began to appear, as
well as dark matter particles. At $10^{-5}$ seconds after the big bang, protons
and neutrons began to form, which would soon become the building blocks for
nuclei.  At about $1$ second after the big bang, neutrinos decoupled, meaning
they could no longer interact with ordinary matter.  This is important because
neutrinos were previously being used to convert protons to neutrons and vice
versa.  This means that at $1$ second after the big bang, this conversion needed
to stop, so the number of protons to neutrons was set, although the neutrons
began to decay quickly, and could only be preserved by forming deuterium
nuclei, which contain $1$ proton and $1$ neutron.  At about $1$ minute after the
big bang, hydrogen and helium nuclei began to form through the process of
nucleosynthesis.  This would continue to occur until the temperature dropped
too low for nuclear fusion to occur.  Nuclei as large as Lithium 7 were able to
be formed in this phase.  Finally, after about $2^{5}$ years after the big
bang, electrons began to combine with nuclei to form neutral atoms like the ones
we see everyday (Lecture 10/10).

%Conclusion
In all, the theory of cosmic inflation alongside with the big bang theory gives
us great insight into the very early universe.  They allow us to model even some
of the earliest events possible to model, and these models help us explain our
observable universe today. From explaining how $\Omega$ is nearly $1$, to
explaining how most of the observable universe is close in temperature.  These
revelations also pose interesting questions such as the possibilities of eternal
inflation and multiverses.


\end{document}
