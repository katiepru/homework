\documentclass[12pt]{article}
\usepackage[margin=1.0in]{geometry}
\usepackage{indentfirst}

\begin{document}

\noindent Kaitlin Poskaitis\\
Assignment 5\\
October 9, 2013
\begin{center}
    \section*{\bf The Very Early Universe}
\end{center}

%Intro

%Problems with big bang - ADD SOURCES
Although the big bang theory seemed to be the most accurate depiction of the
beginning of the universe, it gave rise to a few problems that could not easily
be explained.  The first of these problems is the problem of flatness.
Scientists have strong evidence for our universe having an $\Omega$ value very
close to 1.  However, if the early universe had even a tiny amount of curvature,
that curvature would be amplified greatly by the expansion of the universe. It
seems extremely unlikely that the early universe had no curvature at all, so to
see the universe currently currently having almost no curvature is equally
unlikely.  The second problem that the big bang theory poses is that the
temperature of the observable universe is extremely close to being equivalent
everywhere.  Specifically, the temperature of the sky is the same to a part in
$1.0E^{-5}$, even on scales that should not have been causally connected in the
early universe.  The third issue with the big bang theory is that Grand Unified
Theories
predict that strange phenomena, such as monopoles, should exist.  The Grand
Unified Theories (GUT) refers to when there is enough energy to have the weak
force, the strong force, and the electromagnetic force all converge into one
force, which we believe occurred during the first fractions of a second after the
big bang.

%Solutions!!
While these seem like relatively difficult problems to explain, the theory of
cosmic inflation provides an explanation for them all.  The theory of cosmic
inflation explains how hyperinflation of spacetime occurred during the first few
fractions of a second after the big bang.  This provides a solution to

\end{document}
