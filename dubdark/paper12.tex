\documentclass[12pt]{article}
\usepackage[margin=1.2in]{geometry}
\usepackage{indentfirst}

\begin{document}

\noindent Kaitlin Poskaitis\\
Assignment 12\\
December 5, 2013
\begin{center}
    \section*{Supermassive Black Holes}
\end{center}

Black holes are extremely dense objects that create a singularity in spacetime
because of their immense gravity.  They come in several size categories, but
they all share some basic properties.  First, they have an event horizon, which
describes the area around the black hole where the escape velocity becomes equal
to $c$.  This means that light cannot escape the gravity of the black hole once
it crosses that horizon, which is why inactive black holes cannot be directly
observed.  The event horizon is very small, on the order of parsecs, but shows
how immense the gravity of these objects is.

Another common characteristic among black holes is the black hole's associated
Schwarzschild metric and radius, both coined by Karl Schwarzschild in 1916.
The Schwarzschild metric presents a solution
to Einstein's field equations that describes the gravitation field around the
black hole, assuming that black hole does not have a magnetic field or angular
momentum.  This metric is calculated from the Schwarzschild radius, which
describes the radius from the black hole to the event horizon.  This radius is
approximately 3 times the mass of the object.

Scientists have seen evidence for two major size classes of black holes:
stellar mass black holes and supermassive black holes.  Stellar mass black holes
are on the magnitude of several solar masses in size and are quite common in the
universe.  They are formed when a very massive star collapses at the end of its
lifetime and is too massive to be able to be help up by even neutron degeneracy
pressure.

Supermassive black holes are very different from this, being on the
order of tens of thousands to a few billion solar masses in size.  They tend to
be found at the center of galaxies, but their origins are not well known.  There
are a few theories as to how initial "seeds" for them could have been formed in
the early universe, which could have then merged to form these giants we see
today.  One theory is that these seeds were formed from the collapse of very
early stars.  Astronomers have evidence that these early stars were much larger
than the ones we see today, so this could result in seeds of a few hundred solar
masses.  A core collapse in a star cluster could have a similar effect.  The
issue with these theories is that we see evidence of supermassive black holes
are far back as redshift 7, which would not have given these relatively small
seeds much time to merge.  Another theory that produces larger seeds is that of
the direct collapse of large gas clouds in early protogalaxies.  Because of the
large amounts of gas that could be involved, this could produce seeds raning
from $10^4-10^5$ solar masses in size.  This still may be too small, but is
certainly better than a few hundred solar masses.


\end{document}
