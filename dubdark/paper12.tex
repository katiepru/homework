\documentclass[12pt]{article}
\usepackage[margin=1in]{geometry}
\usepackage{indentfirst}

\begin{document}

\noindent Kaitlin Poskaitis\\
Assignment 12\\
December 5, 2013
\begin{center}
    \section*{Supermassive Black Holes}
\end{center}

Black holes are extremely dense objects that create a singularity in spacetime
because of their immense gravity.  They come in several size categories, but
they all share some basic properties.  First, they have an event horizon, which
describes the area around the black hole where the escape velocity becomes equal
to $c$.  This means that light cannot escape the gravity of the black hole once
it crosses that horizon, which is why inactive black holes cannot be directly
observed.  The event horizon is very small, on the order of parsecs, but shows
how immense the gravity of these objects is.

Another common characteristic among black holes is the black hole's associated
Schwarzschild metric and radius, both coined by Karl Schwarzschild in 1916.
The Schwarzschild metric presents a solution
to Einstein's field equations that describes the gravitation field around the
black hole, assuming that black hole does not have a magnetic field or angular
momentum.  This metric is calculated from the Schwarzschild radius, which
describes the radius from the black hole to the event horizon.  This radius is
approximately 3 times the mass of the object.

Scientists have seen evidence for two major size classes of black holes:
stellar mass black holes and supermassive black holes.  Stellar mass black holes
are on the magnitude of several solar masses in size and are quite common in the
universe.  They are formed when a very massive star collapses at the end of its
lifetime and is too massive to be able to be help up by even neutron degeneracy
pressure.

Supermassive black holes are very different from this, being on the
order of tens of thousands to a few billion solar masses in size.  They tend to
be found at the center of galaxies, but their origins are not well known.  There
are a few theories as to how initial "seeds" for them could have been formed in
the early universe, which could have then merged to form these giants we see
today.  One theory is that these seeds were formed from the collapse of very
early stars.  Astronomers have evidence that these early stars were much larger
than the ones we see today, so this could result in seeds of a few hundred solar
masses.  A core collapse in a star cluster could have a similar effect.  The
issue with these theories is that we see evidence of supermassive black holes
are far back as redshift 7, which would not have given these relatively small
seeds much time to merge.  Another theory that produces larger seeds is that of
the direct collapse of large gas clouds in early protogalaxies.  Because of the
large amounts of gas that could be involved, this could produce seeds raning
from $10^4-10^5$ solar masses in size.  This still may be too small, but is
certainly better than a few hundred solar masses.

Supermassive black holes gain mass primarily through the accretion of
surrounding gas.  While this is a somewhat complex process, it is important in
understanding the physics of these objects.  Essentially, nearby gas is
attracted to the black hole because of its gravity, but its angular momentum
prevents a lot of it from falling inside the black hole.  This process resembles
how many disk shaped objects in the universe form, such as the disks in a spiral
galaxies or the planets orbiting the sun.  This gas settles into a disk shape,
forming what is known as the accretion disk.  Friction causes some of this gas
to fall into the hole, and also produces heat.  As a result, the gas is in a
constant battle between outward radiation pressure and inward gravitational
pull.

The radiation produced by an accreting black hole makes them much easier to
identify because this radiation can be directly measured.  However, it is
important that this radiation be distinguishable from a stellar body giving off
radiation, and this is in fact true.  The main difference between stellar
radiation and black hole radiation is in the associated spectra.  Stars have a
spectra similar to that of black body radiation, with a large peak typically in
the optical range.  Black holes, on the other hand, display large amounts of
radiation coming from most wavelengths.  They are particularly loud in the
x-ray, and also show lines from high-energy ionized gas.  These are very
different from stars, so if a decent spectrum can be recorded, it is possible to
distinguish black holes from stars.

While it is not terribly difficult to observe accreting black holes, it is much
more difficult to observe a dormant supermassive black hole because its presence
can only be inferred from its gravitational effects on other bodies in the
system.  Their presence causes two observable effects on the nearby stars:
increased velocities and distorted orbits.  In a galaxy without a black hole,
the central stars' velocities max out at around 100 km/s, where in galaxies with
a supermassive black hole these velocities are much higher because of the
immense gravity.  Furthermore, the central stars in a galaxy without a
supermassive black hole have a random distribution of orbit shapes, while
galaxies with a black hole tend to have more central stars with cigar-shaped
orbits.  This causes more stars to be near the center of these galaxies at any
given time, forming what is known as the cusp.  Unfortunately, both of these
types of observations can only be made in nearby galaxies because the angular
resolutions of these central stars is so small.

One interesting relationship between supermassive black holes and their host
galaxies is that the size of the central bulge of the galaxy is proportional to
the mass of the black hole.  This was a very surprising correlation because the
black hole should not have been able to directly influence such a large area.
This may to due to galaxy mergers where multiple supermassive black holes merged
and stripped the center of cool gas in a process, although this is still unclear
why.

The energy released from accreting black holes has a significant impact on the
host galaxy through a proccess called AGN feedback.  This radiation heats the
galaxy, causing the area around it to slow star formation because it heats the
cool gas needed for this process.  In this way, the supermassive black hole acts
as a kind of galactic thermostat, heating the galaxy when there is a lot of gas
in the center and shutting off when this gas is gone.  This would explain some
issues astronomers initially had with galactic simulations, such as the fact
that galaxies were expected to be much cooler with more star formation occuring
in the center.  It also explains why there are fewer giant galaxies than
expected because of the prevention of star formation.

In recent years, it has even been discovered that there is a supermassive black
hole in the center of the Milky Way, and it is called Sagittarius A.  It was
discovered when astronomers noticed large x-ray flares coming from its general
area, which is a phenomenon that supermassive black holes often create.  From
the orbits of nearby stars, it has an estimated mass of 4 million solar masses.
Furthermore, it is believed that it was once an AGN because of the presence of
remnants of large radio lobes coming from Sagittarius A.  It is believed to be
accreting at a very low level today, but may become active again in the future.

\end{document}
