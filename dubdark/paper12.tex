\documentclass[12pt]{article}
\usepackage[margin=1.2in]{geometry}
\usepackage{indentfirst}

\begin{document}

\noindent Kaitlin Poskaitis\\
Assignment 12\\
December 5, 2013
\begin{center}
    \section*{Supermassive Black Holes}
\end{center}

Black holes are extremely dense objects that create a singularity in spacetime
because of their immense gravity.  They come in several size categories, but
they all share some basic properties.  First, they have an event horizon, which
describes the area around the black hole where the escape velocity becomes equal
to $c$.  This means that light cannot escape the gravity of the black hole once
it crosses that horizon, which is why inactive black holes cannot be directly
observed.  The event horizon is very small, on the order of parsecs, but shows
how immense the gravity of these objects is.

Another common characteristic among black holes is the black hole's associated
Schwarzschild metric and radius, both coined by Karl Schwarzschild in 1916.
The Schwarzschild metric presents a solution
to Einstein's field equations that describes the gravitation field around the
black hole, assuming that black hole does not have a magnetic field or angular
momentum.  This metric is calculated from the Schwarzschild radius, which
describes the radius from the black hole to the event horizon.  This radius is
approximately 3 times the mass of the object.




\end{document}
