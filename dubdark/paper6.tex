\documentclass[12pt]{article}
\usepackage[margin=1.1in]{geometry}
\usepackage{indentfirst}

\begin{document}

\noindent Kaitlin Poskaitis\\
Assignment 7\\
October 18, 2013
\begin{center}
    \section*{\bf Structure Formation and Gastrophysics}
\end{center}

%Intro

%Lumpy
One somewhat more recent discovery relating to the structure formation on large
scales in our universe is the small density inhomogeneities observed at
temperature differences in the Cosmic Microwave Background at around the time of
recombination (Lecture 10/15).  This observation plays an extremely important
role in helping us
understand how conditions in the early universe led to the formation of large
scale structures seen today.  Essentially, these small differences in density
would cause the areas with higher than average density to pull in matter from
areas with average or below average densities (Coles 2001).  This would cause
a ripple
effect, as these areas would continue to acquire larger and larger amounts of
gas.  Eventually, this would cause filaments and sheets like those seen in
cosmic structure today.  Furthermore, only very small differences in density
could kick off this process, and scientists theorize that these small
differences in density were due to cosmic inflation causing quantum
fluctuations (Coles 2001).

%N-body
The gravitational attraction that took place between these areas of high and low
density can be computationally modelled through a technique called N-body
simulations.  Essentially, these simulations take in a coordinate system
composed of $n$ bodies (in this case dark matter particles) and simulates the
gravitational effects they all have on each other (Lecture 10/15).  This is a
computationally
hard problem, but through the use of massive super computers, scientists have
been able to create these simulations in order to model the conditions of our
early universe and observe the changes over time.  The changes observed are
pretty much in line with actual observation when cold dark matter is used
(Lecture 10/15. %WRONG

%Cold Dark Matter
When applying cold dark matter to large scale structure formation, it is often
referred to as a hierarchical model because of the bottom-up way of forming
structure (Combes 2009).  Specifically, when cold dark matter is used in simulations,
very small structures form first, then merge together to form larger structures,
and so on.  Essentially, this means that the largest of structures in the
universe will have taken the longest to form.  This is the opposite of what
would happen if we used warm dark matter, as in that case we would see a
top-down pattern in the formation of structures.  This means that the largest
structures would form first, then would begin to break down into smaller
structures over time (Combes 2009).

%Violent relaxation
So far we have discussed gravity's effect on the dark matter particles that make
up most of the matter in the universe, but gravity's effect on baryonic
particles also plays a large role in galaxy and general large scale structure
formation, specifically through a process called violent relaxation.
Essentially, this occurs between a gravitationally bound pair consisting of a
dark matter particle and a baryonic particle.  Because of their gravitational
attraction, they will attract each other and eventually come together.  However,
the dark matter particle cannot interact with the baryonic particle in any other
way other than gravitationally so they will fly past each other for a shorter
distance then begin coming together again.  This forms an oscillation pattern
that eventually allows the baryonic particle to come to rest in a gravitational
bind with the dark matter particle.




\end{document}
