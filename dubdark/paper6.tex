\documentclass[12pt]{article}
\usepackage[margin=1.1in]{geometry}
\usepackage{indentfirst}

\begin{document}

\noindent Kaitlin Poskaitis\\
Assignment 7\\
October 18, 2013
\begin{center}
    \section*{\bf Structure Formation and Gastrophysics}
\end{center}

%Intro
When the universe was very young, it was believed that the density was very
close to uniform, which made it extremely difficult to explain how any of the
magnificent large scale structures we see today in our universe could have
formed.  However, once it was found that the density differed slightly at the
time of recombination, suddenly scientists were able to model how these
structures formed.  This is still a very active area of research, and it all
began with an observation about the cosmic microwave background.

%Lumpy
One somewhat more recent discovery relating to the structure formation on large
scales in our universe is the small density inhomogeneities observed at
temperature differences in the Cosmic Microwave Background at around the time of
recombination (Lecture 10/15).  This observation plays an extremely important
role in helping us
understand how conditions in the early universe led to the formation of large
scale structures seen today.  Essentially, these small differences in density
would cause the areas with higher than average density to pull in matter from
areas with average or below average densities (Coles 2001).  This would cause
a ripple
effect, as these areas would continue to acquire larger and larger amounts of
gas.  Eventually, this would cause filaments and sheets like those seen in
cosmic structure today.  Furthermore, only very small differences in density
could kick off this process, and scientists theorize that these small
differences in density were due to cosmic inflation causing quantum
fluctuations (Coles 2001).

%N-body
The gravitational attraction that took place between these areas of high and low
density can be computationally modelled through a technique called N-body
simulations.  Essentially, these simulations take in a coordinate system
composed of $n$ bodies (in this case dark matter particles) and simulates the
gravitational effects they all have on each other (Lecture 10/15).  This is a
computationally
hard problem, but through the use of massive super computers, scientists have
been able to create these simulations in order to model the conditions of our
early universe and observe the changes over time.

%Problems with simulations and such
Although we are able to use N-body simulations to model the large scale
structure formation in our universe, there have been some issues with the
simulations in that they do not fully align with actual observations, even with
cold dark matter and dark energy accounted for (Lecture 10/17).  Specifically,
there are three
main problems that have been noticed.  First, the simulations predict the gas
found in galaxies should be much colder due to the expansion of the universe.
Second, the simulations also predicted that the average angular momentum of
stars in galaxies should have been much lower, therefore resulting in smaller
galaxies.  Finally, the simulations predicted that we should see around 100
dwarf satellite galaxies, but we only see around 40.

%Solutions
Even though these seem like difficult problems, it may be possible that our
miscalculations on all of these are at least partially caused by the same
phenomenon: star formation (Lecture 10/17). Star formation could partially
account for the
temperature of gas in galaxies being hotter than it should be start formation
gives off tremendous amounts of heat.  Only only issue with this is that there
is not a lot of star formation in older galaxies, and the same problem still
exists there.  In large, this is still unsolved.  Star formation could have
caused the angular momentum problem because the simulations had a lot of star
formation going on in small dark matter halos, which would cause them to lose
angular momentum.  If this is an inaccurate distribution of star formation, this
could easily cause the problem.  Lastly, the issue of observing fewer dwarf
galaxies than should exist could be due to stars not forming in all of the dark
matter halos that could lead to dwarf galaxy formation.  If starts did not form
there, then we would not observe a galaxy even if the halos do exist (Lecture
10/17).

%Cold Dark Matter
When applying cold dark matter to large scale structure formation, it is often
referred to as a hierarchical model because of the bottom-up way of forming
structure (Combes 2009).  Specifically, when cold dark matter is used in simulations,
very small structures form first, then merge together to form larger structures,
and so on.  Essentially, this means that the largest of structures in the
universe will have taken the longest to form.  This is the opposite of what
would happen if we used warm dark matter, as in that case we would see a
top-down pattern in the formation of structures.  This means that the largest
structures would form first, then would begin to break down into smaller
structures over time (Combes 2009).

%Violent relaxation
So far we have discussed gravity's effect on the dark matter particles that make
up most of the matter in the universe, but gravity's effect on baryonic
particles also plays a large role in galaxy and general large scale structure
formation, specifically through a process called violent relaxation (Lecture
10/15).
Essentially, this occurs between a gravitationally bound pair consisting of a
dark matter particle and a baryonic particle.  Because of their gravitational
attraction, they will attract each other and eventually come together.  However,
the dark matter particle cannot interact with the baryonic particle in any other
way other than gravitationally so they will fly past each other for a shorter
distance then begin coming together again.  This forms an oscillation pattern
that eventually allows the baryonic particle to come to rest in a gravitational
bind with the dark matter particle (10/15).

%Conclusion
In all, the differences in density of dark matter in the early universe is what
caused the large scale structure we observe today to form.  We can use N-body
simulations to estimate how exactly everything formed, but these still have some
limitations that are currently being researched by astronomers.  These
simulations along with other physical laws have allowed us to predict a
bottom-up hierarchical model of large scale structure formation.  This, which
allowed dark matter to form halos, combined with violent relaxation is what we
believe is the way that early galaxies formed.  Much research still needs to be
done in order to account for some discrepancies, but the cold dark matter model
seems to have scientists going in the right direction with their research.


\end{document}
