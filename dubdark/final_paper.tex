\documentclass[12pt]{article}
\setcounter{secnumdepth}{0}
\usepackage{hanging}
\usepackage[margin=1in]{geometry}
%\usepackage{indentfirst}

\title{Growth of Supermassive Black Holes and Scaling Relations
              Between Supermassive and Host Galaxies}
\date{}
\author{Kaitlin Poskaitis}

\begin{document}

\clearpage
\maketitle
\thispagestyle{empty}

\tableofcontents

\pagebreak
\setcounter{page}{1}

Not long ago, supermassive black holes were a completely separate field of study
from that of galaxies.  However, more recent discoveries have shown that these
two types of astronomical objects are very closely related and have measurable
effects on each other.  There are numerous of these effects, and some of these
have been able to explain several of the pitfalls of our previous N-body
simulations, such as having fewer large galaxies than expected as well as
galaxies remaining hotter, and therefore less star formation occurring in older
galaxies than expected.  Therefore, one can see why it would be important to
understand the scaling relations and resulting co-evolution of supermassive
black holes and their host galaxies.

\subsection{Black Hole Basics}
Black holes are extremely massive and dense objects that form a singularity in
spacetime
because they are so massive.  They are so massive that they have what is called
an event horizon, or a position where the escape velocity of any particle
becomes greater than the speed of light ($c$).  This means that even light cannot
escape the black
hole's gravity, which is why they cannot be directly observed.  The event
horizon exists because general relativity states that spacetime will become
infinitely curved near a singularity, making it impossible to escape from
without moving faster than $c$.
In 1916, an astronomer named Karl Schwarzschild described the gravitation field
around a spherical mass such as a black hole.  This is known as the
Schwarzschild metric, and is a solution to Einstein's field equations that
assumes the mass has no charge or angular momentum.  This metric is dependent on
the Schwarzschild radius, which is the distance from the spherical mass to its
event horizon.  For example, the Schwarzschild radius of the earth is only a few
millimeters, whereas the Schwarzschild radius of any black hole
approximately $3M$ kilometers, where $M$ is the mass of the black hole in solar
masses.  This relationship was discovered by Karl Schwarzschild in 1916.

Black holes are believed to be able to exist in three different size classes:
miniature, stellar mass, and supermassive.  Stellar mass black holes are the
most common and are formed from the implosion of a very massive dying star.  If
the star is too massive to be supported by electron or neutron degeneracy
pressure, then it becomes a black hole.  There is a
great deal of observational evidence for these stellar mass black holes, and
they are most commonly detected in binary systems with other stars.  The black
holes are inferred to be there because of their strong gravitational effects,
which would increase the velocity of the other star in the system and can even
cause some gas to be pulled off of the other star and onto the black hole.
Another type of black hole is the miniature black hole.  Miniature
black holes have not been observed, but are theoretically possible to be as
small as particles.  Supermassive black holes are much larger than stellar mass
black holes, on average $10^6 -10^9$ solar masses in size.

\subsection{Supermassive Black Hole Properties and Formation}
Supermassive black holes retain the basic properties of stellar mass black holes
but are scaled up to enormous masses.  This means that they continue to be
singularities that cause infinite curvature in spacetime because of their huge
densities.  Scientists know that these black holes primarily grow from accreting
matter (this process is described in detail below), and by merging with other
black holes.  However, scientists do not know what causes these objects to form
in the first place.  A few theories for their formation have been proposed.  For
example, it may be possible that large "seed" black holes were formed when early
stars collapsed, as they were hundreds of solar mases in size.  These large
black holes could have then merged to form the supermassive black holes we see
today.  Another proposed solution to this problem is that they formed from the
core collapse of large stars clusters, but again this would only leave black
holes a few hundreds of solar masses in size at most.  A proposed solution that
predicts the formation of larger black holes initially is the direct collapse of
huge gas clouds in early protoglaxies, which would result in black holes between
$10^4 - 10^5$ solar masses in size.  This seems to be the most likely as of now
because we can see supermassive black holes as far back in time as redshift 7,
which would not give these objects a lot of time to form.

Astronomers classify supermassive black holes into two major groups: active and
dormant/quiescent.  The dormant supermassive black holes can only be detected
through
their gravitational effects and are not accreting substantial amounts of matter.
Active supermassive black holes, on the other hand, are accreting large amounts
of matter and as a result are often referred to as active galactic nuclei (AGN).
The details of the accretion process are described in the next section, but some
subtle differences in the observed accretion gives scientists ways to classify
AGN into several subclasses, the main ones being quasars, Seyfert galaxies, and
radio galaxies.  Astronomers can begin to categorize AGN based on their
luminosity in the UV, optical, and infrared spectra.  Quasars fall into the
category where their luminosities in this range are greater than or equal to
that of their host galaxies ($L_{nuc} \ge L_{gal}$).  An AGN is said to be
strong if $L_{nuc} \le L_{gal}$, and it is said to be weak if $L_{nuc} \ll
L_{gal}$.  Seyfert galaxies tend to be strong, while radio galaxies tend to be
weak, although they are more variable than the other types in this regard.  The
other main way to classify an AGN is whether it is radio loud or radio quiet.
Quasars can be radio loud or radio quiet, radio galaxies are always radio loud,
and Seyfert galaxies are always radio quiet.  More information on how these
different AGN can be detected is decribed later.


\subsection{Supermassive Black Hole Accretion}
(CITE science.pdf)
Supermassive black holes grow by accreting large amounts of matter in
galactic systems.  Because of this, an accretion disk forms around the black
hole, and this disk can vary in size.  The formation of this disk is similar
to how any disk-shaped structure forms in the universe, such as the disks in
spiral galaxies or the disk of planets orbiting our sun.  This occurs
because the gas almost always has some angular momentum that prevents it
from directly reaching the gravitational source, in this case the
supermassive black hole.  As a result, the gas settles into a disk-shaped
structure, therefore forming the accretion disk.  This disk's orientation is
defined by the overall angular momentum of the gas.  Given such angular
momentum, it appears that friction must be responsible for some of the gas
losing enough angular momentum to actually fall into the black hole, and it
is most likely magnetic forces in the ionized plasma that provides this
friction.

This same magnetic stress that causes changes in angular momentum also
converts some of the gravitational potential energy of the gas into heat.
The process by which this happens is not well understood, but the details of
this process are not needed to determine its effects on the accretion disk,
as the properties of the accretion disk are largely dependent on whether or
not this produced heat is radiated away.  The radiative efficiency of an
accretion disk is modelled by the following equation:
$\epsilon = \frac{L}{Mc^2}$, where $\epsilon$ is the radiated efficiency
(out of 1), $L$ is the produced luminosity, $M$ is the mass-accretion rate,
and $c$ is the speed of light.  This implies that $Mc^2$ is the rate at
which the mass-energy is accreted.  The Eddington limit, which is the maximum
luminosity the black hole can have while in a state of equilibrium between
gravitation forces and radiative forces, also applies to the
accretion rate as it sets the upper bound for the black hole's luminosity.

Accretion disks can be divided into two
distinct categories based on their corresponding $\epsilon$.  If the
previously mentioned heat produced is radiated away faster than it
takes for the gas to flow back into the hole, the result is a much more
efficient accretion disk, with $0.06 \le \epsilon \le 0.4$.  This results in
a thin disk with a thickness of 0.1 - 3\% of its radius.  The disk is thin
because the gas cools rapidly causing it to settle into a thin disk.  These
thin disks are the most efficient known source of power in the universe,
being over 50 times more efficient than nuclear fusion in stars.  When the
conditions for creating a thin disk are not met, meaning that it takes more
time for the heat to be radiated away than for the gas to flow back into the
hole, then there is a substantial drop in efficiency down to $\epsilon \ll
1$.  This produces a thick, hotter disk that is more than 20\% as thick as
its radius.  These thick disks tend to be seen when the accretion rate $M
\ll M_{edd}$ (Eddington limit).  This is because only a small fraction of
the gas can reach
the hole at any one time, resulting in a much dimmer black hole.  Thick disks
are also preferred when $M \gg M_{edd}$ because the large amounts of gas
prevent heat from being radiated away.  It is only when $M \approx M_{edd}$
that the thin disks are preferred.

The dynamics of the accretion disk seem to play a large role in the dynamics of
another feature of some accreting supermassive black holes: the relativistic
radio jets.  These jets are not present in all supermassive black holes, but
some produce huge radio jets perpendicular to the accretion disk.  These jets
seem to be the most pronounced when the accretion disk is thick, although the
explanation for this phenomenon has not been discovered.  Regardless, these
outflows are much more energetic when the black hole is rotating.  One explanation
for the formation of these jets relies on the properties of the ergosphere,
which is an area of spacetime just outside the event horizon where matter cannot
be at rest and must rotate along with the black hole.  The plasma in this area
gets twisted and propagates out, causing a large outflow.

\subsection{Detecting Supermassive Black Holes}
Because supermassive black holes can be active or quiescent, there are several
different ways to detect them based on the state they are in.  Quiescent
supermassive black holes can only be detected by their gravity's influence on
nearby stars and other objects, while active galactic nuclei are generally
detected by their spectral emission lines and other visible properties.

    \subsubsection{Quiescent Supermassive Black Holes}
    If a supermassive black hole is truly dormant, then it can only be detected
    by its gravitational effects on surrounding visible matter.  Because of
    this dilemma, the process of detecting dormant supermassive black holes can
    be compared to that of stellar mass black holes.  Recall that stellar mass
    black holes are most often detected when in binary systems because of their
    effects on the other star in the system.  A similar concept applies to
    supermassive black holes, only instead of being in a binary system, their
    system consists of all of the central stars in a galaxy.  Obviously this
    complicates the process, but there are still some very distinct effects that
    can be observed, all of which are heavily dependent on the black hole's
    sphere of influence.  This describes the area in which the black hole's
    gravity completely overwhelms that of all of the other stars in the system.
    This is typically a relatively small area, only a few arcseconds in size,
    but is still orders of magnitude larger then the event horizon, meaning that
    objects can be observed in a large portion of it.

    The sphere of influence is so important because the supermassive black
    hole's gravity changes the movement of the central stars in a galaxy in
    several ways.  First, it increases the velocity of stars as they approach the
    black hole.  In a galaxy without a black hole center, the speed of the
    central stars tends to level out at about 100 km/s.  This is not the case in
    galaxies with dormant nuclei, as these speeds would continue to increase
    to larger fractions of $c$.

    Another way that a supermassive black hole's gravity affects central stars
    is by distorting their orbits.  These central stars would normally have a
    random distribution of orbit shapes.  However, when a black hole is
    present, it causes the stars to have more cigar-shaped orbits, especially
    inside the sphere of influence.  This may seem like a subtle distinction,
    but leads to a very important phenomenon: more stars will be near the center
    of the galaxy at any one time, causing the center to appear much more
    luminous.  This group of stars nearer to the center is called the cusp, and
    has been used to successfully detect quiescent supermassive black holes.

    \subsubsection{Active Galactic Nuclei}
    Active galactic nuclei can be easier to detect because they are sources of
    strong radiation, meaning that there are other more direct ways to detect
    them besides their gravitational influence on other objects.  However,
    scientists must be able to distinguish the different types of AGN from each
    other as well as from stellar sources of radiation.  While these
    distinctions are different for each type of AGN, studying an object's
    emission spectra is how these distinctions are made.  Normal stars tend to
    have spectra resembling that of black body radiation, with their peaks being
    in or near the visible range.  Most AGN tend to have much flatter spectra,
    and are especially loud in the x-ray.  They also tend to have at least some
    emission lines indicating higher levels of ionized gas.  These differences are
    typically how AGN are differentiated from stellar radiation sources.

    Once it is clear that a source of radiation is an AGN, there are several
    characteristic features of each of the major types that allow astronomers to
    classify them.  There seem to be two clear ways that AGN can differ from
    each other.  Quasars, for example, can be radio quiet or radio loud
    depending on the type, but tend to have extremely high optical luminosities
    that often outshine the entire host galaxy.  They also tend to
    have broad and narrow lines of high ionization, although rarely the broad
    lines are not present.  Seyfert galaxies, on the other hand, have more
    moderate luminosity and are always radio quiet.  Most have
    broad emission lines, and all have narrow lines of high ionization.  Radio
    galaxies are the most radio loud, but do not have clear patterns in the
    optical spectrum.

    \subsubsection{Observation in the Milky Way}
    Recent research has found a moderately size supermassive black hole at the
    center of the Milky Way.  It is called Sagittarius A, and is believed to be
    around four million solar masses in size.  It was discovered because it
    produced a huge x-ray flare that is characteristic of a supermassive black
    hole.  Scientists also found the remnants of large radio lobes around it,
    further confirming its existence as a supermassive black hole.  This is an
    extremely important discovery because never before have we been able to
    observe the effects of such an object so closely.  We can even track the
    orbits of the central stars around it, which seem to clearly show the
    characteristic cigar-shape orbit patterns and increased velocities.
    Furthermore, its relativistic effects can now be measured up close, which
    will be able to confirm or reject Einstein's theory of general relativity.
    We have never been able to observe these Einstein's theory in action at such
    a huge scale before,
    so it will be a great test of this.

\subsection{Galaxy Properties and Formation}
Astronomers have discovered many different galaxies throughout time, and these
galaxies can be classified in several different ways based on their different
properties.  When considering nearby galaxies (redshift $z < 1$), the Hubble
classification system works well for classifying the many different galaxies in
the Local Group and beyond (CITE hubble classifications).  Hubble's
classification system, which has been revised several times since its initial
incarnation, categorizes galaxies into three main types: spiral, elliptical, and
irregular.  These categories have several subcategories within them describing
the specific morphology of each type, but galaxies that fall into any of these
three categories have many similar characteristics.

    \subsubsection{Spiral Galaxies}
    Spiral galaxies tend to be the go-to example for describing galaxies in general
    as the Milky Way is this type of galaxy.  Spiral galaxies have several main
    morphological components, one of which is the stellar halo (CITE Lecture 10/29).
    This is a spherical region surrounding the galaxy that is on average
    approximately 100 kpcs in size.  This is much larger than the other, more
    luminous visible components.  However, the stars that care found here are very
    sparse and tend to be old, metal-poor stars.  This is also the most common place
    for globular clusters to form.

    Another component of spiral galaxies is the central bulge, which is the very
    luminous central portion of the galaxy which forms a noticeable bulge in the
    center of the disk.  This section contains mostly old, metal-rich stars that
    follow more random orbits due to their proximity to the center.  The central
    bulge on average contains about $\frac{1}{4}$ of the mass of the entire
    galaxy, although the size of the central bulge can vary greatly between
    galaxies.  It also sometimes contains a bar of stars that is more flat.
    This bar can be very pronounced in some galaxies and seemingly nonexistent
    in others, and is one of the ways Hubble determines subcategories within the
    spiral class.

    This central bulge forms the center of the two largest structures in spiral
    galaxies: the thin and thick disks.  These disks make up most of the mass of
    the galaxy, and contain different types of stars in each.  The thin disk,
    which is much larger, contains younger, more metal-rich stars and the thick
    disk contains older, more metal-poor stars.  The stars in these disks have
    higher angular momentums and therefore do not follow random orbits like the
    ones in the central bulge.  Finally, on the edges of the disks, spiral arms
    can be observed, which contain young, metal-rich stars as well as gas and
    dust.

    As a whole, spiral galaxies tend to look very flat because of the thin and
    thick disks.  Furthermore, they tend to have a relatively high amount of new
    star formation taking place because of the size and environment of the thin
    disk.  As a result, spiral galaxies tend to look bluer in color, with a more
    red center as less star formation is occurring there.

    \subsubsection{Elliptical Galaxies}
    Elliptical galaxies are much simpler in morphology than spiral galaxies are.
    They are elliptical in shape, and can be categorized by their ellipticity.
    These galaxies in some ways resemble the central bulge of spiral galaxies in
    that there is very little star formation occurring, leaving this type of
    galaxies populated with older, metal-rich stars.  The stars do not have much
    angular momentum, so also take on random orbits.  Elliptical galaxies tend
    to be more luminous towards their centers, and the luminosity drops farther
    away from the center in a smooth way.  These galaxies tend to appear redder
    in color because of their higher density of older stars and lack of new star
    formation.

    \subsubsection{Irregular Galaxies}
    Irregular galaxies are those galaxies which do not fit into either of these
    classifications.  These galaxies vary greatly from one another, and it is
    therefore difficult to make any generalizations about them.

    Note that these classifications of galaxies do not hold up as well at higher
    redshifts, but it is believed that some of the galaxies we see farther away
    are precursors to these types of galaxies.

    \subsubsection{Galaxy Lifetime}
    In order to fully understand how supermassive black holes affect a host
    galaxy throughout its lifetime, it is important to understand the major
    events that happen throughout a galaxy's lifetime.  The first major
    milestone in a galaxy's life is its formation.  This process is believed to
    be very different for spiral and elliptical galaxies.  For spiral galaxies,
    it is believed that the formation of their basic structure is caused by
    the monolithic collapse of primordial cold dust clouds.  This collapse would
    create an environment where star formation would flourish, and the intrinsic
    angular momentum of the gas would be transferred to new stars.

    On the other
    hand, it is believed that elliptical galaxies form from the merging of two
    spiral galaxies that are similar in size.  This type of merging would
    annihilate the disks, giving the resulting galaxy an elliptical shape.  If
    one galaxy was much bigger than the other, then the larger galaxy would be
    able to acquire the smaller one and still keep its general spiral
    morphology, although some of its features could change such as the size of
    the central bulge.

    These two events, formation and merging, make up the two major events in a
    galaxy's lifetime that dramatically change its morphology.  Supermassive
    black holes are believed to play significant roles in these processes, which
    will be further described below.


\subsection{Scaling Relations}
Supermassive black holes, being incredibly dense objects at the center of some
galaxies, should have some effects on the general morphology of the galaxy. It
has been shown that the size of the central bulge in galaxies is correlated
extremely closely with the mass of the galaxy's supermassive black hole (Brennan
2012).  This
relation is known as the $M \sigma$ relation, where $M$ is the mass of the
central dark object and $\sigma$ is the velocity dispersion of the central
bulge, which is indicative of the mass of the central bulge.  This relationship
has shown that $M \propto \sigma^{4.8}$, and this
relationship has been plotted in Figure 1.  The important takeaway from this
relationship is that the size of the central bulge is proportional to the mass
of the galaxy's supermassive black hole.  Furthermore, the plot shows that there
is very little scatter in the collected data, meaning that this is a very tight
relationship that holds well in most cases (Brennan 2012).  Noting that the bulge
being
referenced refers to the central bulge in spiral galaxies, but the entire
ellipse in elliptical galaxies, it is evident why this relationship is so
important.  The size of elliptical galaxies is often vastly larger than that of
the central bulge in spiral galaxies, meaning that the supermassive black holes
in elliptical galaxies should be much larger than those in spirals.  This is a
logical conclusion from the theory that elliptical galaxies form from the
merging of two or more spirals, as the supermassive black holes at the center of
the spirals would merge into one much larger hole in an elliptical galaxy.

\subsection{Active Galactic Nucleus Feedback}
Since an AGN releases large amounts of radiation, this radiation can have
significant impacts on the host galaxy (Brennan 2012).  This phenomenon when
radiation from an
AGN affects the host galaxy is called AGN feedback.  Because of a phenomenon
called
hydrostatic equilibrium, the radiation pressure from the AGN cannot overpower
its gravity or else it would blow itself apart. This is the basis for the
Eddington limit, as described above.  Because it is possible for AGN to exceed
this limit based on the size and characteristics of the
accretion disk, it is possible that these AGN could have removed a substantial
amount of gas from early prototgalaxies.  This could be possible even with a black
hole $10^7$
solar masses in size from the AGN feedback.  There are several different methods
that the AGN facilitate feedback with, each of which can have different effects on
the host galaxy.  In reality, most AGN would likely utilize more than one of
these methods below (Brennan 2012).

    \subsubsection{Thermally Driven Winds}
    One simple way that AGN feedback could occur is by the produced radiation
    heating the surrounding gas (Brennan 2012).  If the gas is heated enough,
    it will have
    enough energy to escape the black hole's gravitational field.  When this gas
    escapes, it creates an observable wind that is thermally driven.  These
    winds tend to be low in velocity, and this is most likely because this
    method of feedback would be most effective for the gas farthest away from
    the black hole as it would have the lowest escape velocity.  The gas that is
    closest to the black hole would need to be heated to much higher
    temperatures before it could escape entirely.  Even so, heating the gas
    below the escape velocity could have substantial impacts on the accretion
    disk even though the gas would not be able to escape.  This would
    effectively cause the accretion disk to "puff up", allowing it to cool more
    effectively and achieve a balance (Brennan 2012).

    \subsubsection{Magnetically Driven Winds}
    Magnetically driven winds are likely important sources of outflow, but
    the exact mechanism behind them is not well understood (Brennan 2012).
    From what we do
    know, this process requires a magnetic field that can penetrate the
    accretion disk, and that close to the center of the disk the magnetic field
    is bound to move with the gas in the disk.  As the black hole pulls in new
    material, the magnetic lines therefore get kinked.  As a result of this, the
    orbit of the accretion disk becomes unstable at an angle greater than 30
    degrees, resulting in material being blown away along the now-outward
    magnetic field lines.  Therefore, this magnetic field could become a driver
    for outflows given these initial conditions were met (Brennan 2012).

    \subsubsection{Radiation Pressure}
    Another way by which an AGN create outflows is through the radiation
    pressure of photons on
    the surrounding material (Brennan 2012).  However, if the surrounding material
    is fully
    ionized, the AGN's luminosity would have to be much higher than the
    Eddington limit, which we know is not the case in most AGN.  This means that
    the gas cannot be fully ionized to create substantial outflows.  If there is
    a fair amount of neutral dust around the AGN and we look at a larger area
    around it, the effective Eddington limit often increases by a factor of
    1000.  This would cause the neutral dust to be pushed out.  Furthermore,
    this dust would tend to be charged in this type of environment, so it could
    couple with ionized gas therefore driving out a lot of ionized gas along
    with it.
    Through this process, it is shown how radiation pressure can cause
    significant outflows that would affect the host galaxy (Brennan 2012).

    \subsubsection{Kinetic Feedback}
    All of the previously mentioned methods of AGN feedback have been based on
    radiation, but there is kinetic feedback as well (Brennan 2012).  There is
    one main cause
    of this kinetic feedback: the relativistic jets and their resulting
    radio lobes along
    with x-ray cavities.  The relativistic jets are very large,
    especially in radio galaxies.  They often extend into the inner few parsecs
    of the galaxy and are generated by the AGN.  While their inner workings are
    still an open topic of research, they seem to have had different effects on
    the host galaxies at different times in cosmic history.  For example, at
    large redshifts ($z \ge 2$) cold gas was more abundant and these jets would
    have triggered shocks that could thereby stimulate star formation.  However,
    in the more recent universe ($z \le 1$) more of the cold gas has already
    been depleted, so these jets only prevent cooling flows from entering the
    system, therefore inhibiting star formation (Brennan 2012).

    These jets, being thin and directional, heat the entire region around the AGN
    (Brennan 2012).
    This heating causes large bubbles that are visible in the x-ray.  These are
    sometimes called x-ray cavities because the bubbles are pushing away x-ray
    emitting material.  They are also sometimes called radio bubbles, because in
    radio galaxies these line up very well with the radio lobes, as shown in
    Figure 2.  One possible explanation for these bubbles is that the jets are
    moving very fast, so they build up material in front of them until they can
    no
    longer move without being impeded.  This shocks the material in front of
    them, causing them to expand which also lowers their densities.  Once the
    jets "turn
    off", these bubbles become buoyant and rise because of the lowered density.
    This is the best model we have currently to describe this phenomenon
    (Brennan 2012).

\subsection{Effects of AGN Feedback}
AGN feedback is a complicated mechanism that can have some profound effects on
the host galaxy, and can even affect nearby galaxies in the same cluster, albeit
in very different ways (McNamara 2012).  Both of these scales are remarkable
considering how
small a supermassive black hole is in size and how small its sphere of influence
is.  However, the apparent effects of AGN feedback have even been able to solve
some previous problems in modelling galaxy formation and evolution, so it is
clearly an important process (McNamara 2012).

    \subsubsection{Effects on Host Galaxy}
    One of the simplest ways in which AGN affect their host galaxies is by
    producing large amounts of radiation that heats the center of the galaxy
    (McNamara 2012).
    This inhibits star formation by making the gas too hot to form stars
    in the center of these galaxies.  This also explains why the central bulge
    of spiral galaxies is proportional to the size of the supermassive black
    hole: these black holes need to accrete matter as their primary source of
    growth, so this could imply that a larger amount of gas has been heated or
    blown out of the center of galaxies with larger galactic nuclei, making
    these regions devoid of new star formation.  This is typically what is
    observed in the center of spiral galaxies, so this lines up well with
    predictions (McNamara 2012).

    Galactic nuclei also seem to participate in a feedback loop regarding their
    levels of accretion (McNamara 2012).  While this process is also not well
    understood, in a
    general sense they will accrete while there is a large amount of gas and
    dust to form an accretion disk.  As the produced radiation heats this disk,
    a lot of gas is caught in an outflow of some sort, so it eventually does not
    have enough fuel to continue accreting at such a high rate.  This allows the
    area around the gas to cool again because of the reduced radiation, so it
    can then be accreted again, and so on.  This means that a supermassive
    black hole can act as a sort of galactic thermostat, so they are able to
    heat galaxies to a certain point, but cannot make them too hot without
    losing their fuel source (McNamara 2012).

    This feedback loop explains some issues with our previous simulations on
    galaxy formation and evolution (McNamara 2012).  First, the simulations
    were showing
    galaxies cooling down much more quickly than they apparently do.  This can
    be explained by the AGN heating the galaxies to a certain point.
    Furthermore, previous simulations had also predicted too many large
    galaxies.  The heating process can also explain this as it prevents a lot of
    star formation in the center of the galaxy (McNamara 2012).

    \subsubsection{Effects on Clusters}
    Observational evidence has led scientist to believe that AGN also have some
    significant effects on entire galaxy clusters as well (McNamara 2012).
    This was at first
    seen as an absurd notion, as it seemed impossible that such a small object
    could show influence on one of the largest scale structures in the universe.
    Even so, AGN seem to have a similar effect on galaxy clusters as they do on
    their individual host galaxies: they limit gas cooling and inhibit star
    formation at the core of these clusters.  This is because their large
    outbursts that are seen in the x-ray displaces gas, drives shock and sound
    waves, and transports low-entropy, metal-rich gas out from the core of the
    cluster.  A similar feedback loop is also in place here, as it is very rare
    to see these AGN using more power than the minimum to produce these effects.
    These effects are also the strongest in clusters with the most AGN, showing
    a similar relationship to that of the size of the central bulge in spiral
    galaxies (McNamara 2012).

\subsection{Supermassive Black Holes and Galaxy Mergers}
It is evident that AGN have significant effects on their host galaxies and host
galaxy clusters during these structures' lifetimes.  These AGN change the
morphology of their hosts slightly, although this morphological change is
generally restricted to the center of the host.  However, as mentioned earlier,
one of the biggest morphological changes a galaxy can undergo in its lifetime is
the event of merging with another galaxy.  While it was previously believed that
the presence of supermassive black holes did not play a large role in galaxy
mergers, the opposite is now believed to be true. (Khan et al. 2012)

When two galaxies with supermassive black holes merge, if they were both
rich with gas and dust, the black holes will be able to accrete as more gas and
dust
is brought near them (Khan et al. 2012).  This causes the area around the black
holes to
heat up more so than
it would from just the friction that would still be present without the black
holes.  One of two outcomes can occur from this phenomenon, which depend on how
much heat is produced.  If enough heat is produced, as the black holes merge
they will blow away most of the gas and dust in the system, resulting in a
galaxy that is not suitable for star formation, specifically an elliptical
galaxy.  If not enough heat is produced, then the resulting galaxy remains
spiral because the gas and dust is not blown away, but the central bulge will be
much bigger.  The amount of heat produced typically has to do with the relative
sizes of the galaxies: more evenly-sized galaxies tend to produce enough heat in
this process to become ellipticals.  This follows the same trend that the stars
in the galaxy follow regarding whether they form elliptical or a larger spiral
galaxy (Khan et al. 2012).

If one or both galaxies in a merger are gas and dust poor, a very different
picture emerges (Khan et al. 2012).  This means that at least one galaxy was
elliptical or
irregular, so the morphology of the resulting galaxy will be similar.  However,
the galactic nucleus of the resulting galaxy will be quite different: is has
been theorized that these types of mergers result in a binary system of
supermassive black holes.  This would fling any nearby stars out of the system
and create a very powerful gravitational source as the new galactic nucleus.
There has not been any observational evidence for this occurring yet, but it is
theoretically possible and a probably outcome of this type of merger
(Khan et al. 2012).

\subsection{Conclusion}
In all, supermassive black holes as galactic nuclei can have some very
pronounced effects on scales that seem incredibly larger for their relative
size.  The high levels of accretion seen by AGN works in heating the host galaxy
and host cluster and acts as a galactic thermostat.  The radiation produced from
this process can cause huge outflows of gas from the center of the system which
we are able to observe in the x-ray and radio ranges.  These phenomena have even
been able to help scientists understand why previous models of galaxy formation
fell short of empirical observation, and new models now more accurately reflect
the observations of our universe.  Finally, they also play a huge role in galaxy
mergers and help explain a lot of the correlations we see when observing
galaxies, such as ellipticals having such little star formation.  Further
understanding of these cosmic beasts will allow us to better model the universe
we live in and put general relativity to the test, which shows their worth in
scientific research.

\pagebreak
\subsection{References}

\begin{hangparas}{.25in}{1}

Brennan, R., 2012, The Effect of Active Galactic Nuclei on their Host Galaxies,
Rachel Somerville (November 26, 2013).

\vspace{5mm}
Ferrarese, L., Merritt, D., 2002, Supermassive black holes, Physics World, June
2002.

\vspace{5mm}

Heckman, T. M., 2008, The Co-Evolution of Galaxies and Black Holes: Current
Status and Future Prospects, Cornell University Library,
http://arxiv.org/abs/0809.1101, (October 17, 2013).

\vspace{5mm}

Kauffmann, G., van den Bosch, F., 2002, The life cycle of galaxies, Scientific
American, June 2002, p.48-58.

\vspace{5mm}

Khan, F., M., Preto, M., Berentzen, I., Just, A., Spurzem, R., 2012, Mergers of
Unequal Mass Galaxies: Supermassive Black Hole Binary Evolution and Structure of
Merger Remnants, Cornell University Library, http://arxiv.org/abs/1202.2124v1
(December 5, 2013).

\vspace{5mm}

McConell, N. J., Ma, C., 2012, Revisiting the Scaling Relations of Black Hole
Masses and Host Galaxy Properties, Cornell University Library,
http://arxiv.org/abs/1211.2816 (October 13, 2013).

\vspace{5mm}

McNamara, B., R., Nulsen, P., E., J., 2012, Mechanical feedback from active
galactic nuclei in galaxies, groups and clusters, {\it New Journal of Physics},
v14-5.

\end{hangparas}


\end{document}
