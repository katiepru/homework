\documentclass[12pt]{article}
\usepackage[margin=1in]{geometry}
%\usepackage{indentfirst}

\title{Growth of Supermassive Black Holes and Scaling Relations
              Between Supermassive and Host Galaxies}
\date{}
\author{Kaitlin Poskaitis}

\begin{document}

\maketitle

Not long ago, supermassive black holes were a completely separate field of study
from that of galaxies.  However, more recent discoveries have shown that these
two types of astronomical objects are very closely related and have measurable
effects on each other.  There are numerous of these effects, and some of these
have been able to explain several of the pitfalls of our previous N-body
simulations, such as having fewer large galaxies than expected as well as
galaxies remaining hotter, and therefore less star formation occurring in older
galaxies than expected.  Therefore, one can see why it would be important to
understand the scaling relations and resulting co-evolution of supermassive
black holes and their host galaxies.

\subsection*{Black Hole Basics}
Black holes are extremely massive and dense objects that form a singularity
because they are so massive.  They are so massive that they have what is called
an event horizon, or a position where the escape velocity of any particle
becomes greater than $c$.  This means that even light cannot escape the black
hole's gravity, which is why they cannot be directly observed.  The event
horizon exists because general relativity states that spacetime will become
infinitely curved near a singularity, making it impossible to escape from
without moving faster than the $c$.  The distance
from the singularity to the event horizon is called the Schwartzchild radius,
and is approximately $3M$ where $M$ is the mass of the black hole in solar
masses.  This relationship was discovered by Karl Schwartzchild in 1916.

Black holes are believed to be able to exist in three different size classes:
miniature, stellar mass, and supermassive.  Stellar mass black holes are the
most common and are formed from the implosion of a dying star, which occurs when
the star is too massive to even be able to form a neutron star.  There is a
great deal of observational evidence for these stellar mass black holes, and
they are most commonly detected in binary systems with other stars.

\subsection*{Supermassive Black Hole Properties and Formation}
Supermassive black holes are essentially scaled-up stellar mass black holes.

\subsection*{Detecting Supermassive Black Holes}
There are different ways to detect different types of active galactic nuclei.

    \subsubsection*{Quiescent Supermassive Black Holes}
    If a supermassive black hole is truly dormant, then it can only be detected
    by its gravitational effects on surrounding visible matter.

    \subsubsection*{Quasars}
    Quasars need to be distinguished from stellar sources of radiation since
    they look very similar as they are both point sources of radiation.

    \subsubsection*{Seyfert Galaxies}

    \subsubsection*{Radio Galaxies}

\subsection*{\bf Galaxy Properties and Formation}
Astronomers have discovered many different galaxies throughout times, and these
galaxies can be classified in several different ways based on their different
properties.  When considering nearby galaxies $z < 1$, the Hubble
classification system works well for classifying the many different galaxies in
the Local Group and beyond (CITE hubble classifications).  Hubble's
classification system, which has been revised several times since its initial
incarnation, categorizes galaxies into three main type: spiral, elliptical, and
irregular.  These categories have several subcategories within them describing
the specific morphology of each type, but galaxies that fall into any of these
three categories have many similar characteristics.

    \subsubsection*{Spiral Galaxies}
    Spiral galaxies tend to be the go-to example for describing galaxies in general
    as the Milky Way is this type of galaxy.  Spiral galaxies have several main
    morphological components, one of which is the stellar halo (CITE Lecture 10/29).
    This is a spherical region surrounding the galaxy that is on average
    approximately 100 kpcs in size.  This is much larger than the other, more
    luminous visible components.  However, the stars that care found here are very
    sparse and tend to be old, metal-poor stars.  This is also the most common place
    for globular clusters to form.

    Another component of spiral galaxies is the central bulge, which is the very
    luminous central portion of the galaxy which forms a noticeable bulge in the
    center of the disk.  This section contains mostly old, metal-rich stars that
    follow more random orbits due to their proximity to the center.  The central
    bulge on average contains about $\frac{1}{4}$ of the mass of the entire
    galaxy, although the size of the central bulge can vary greatly between
    galaxies.  It also sometimes contains a bar of stars that is more flat.
    This bar can be very pronounced in some galaxies and seemingly nonexistent
    in others, and is one of the ways Hubble determines subcategories within the
    spiral class.

    This central bulge forms the center of the two largest structures in spiral
    galaxies: the thin and thick disks.  These disks make up most of the mass of
    the galaxy, and contain different types of stars in each.  The thin disk,
    which is much larger, contains younger, more metal-rich stars and the thick
    disk contains older, more metal-poor stars.  The stars in these disks have
    higher angular momentums and therefore do not follow random orbits like the
    ones in the central bulge.  Finally, on the edges of the disks, spiral arms
    can be observed, which contain young, metal-rich stars as well as gas and
    dust.

    As a whole, spiral galaxies tend to look very flat because of the thin and
    thick disks.  Furthermore, they tend to have a relatively high amount of new
    star formation going on because of the size and environment of the thin
    disk.  As a result, spiral galaxies tend to look bluer in color, with a more
    red center as less star formation is occurring there.

    \subsubsection*{Elliptical Galaxies}
    Elliptical galaxies are much simpler in morphology than spiral galaxies are.
    They are elliptical in shape, and can be categorized by their ellipticity.
    These galaxies in some ways resemble the central bulge of spiral galaxies in
    that there is very little star formation occurring, leaving these type of
    galaxies populated with older, metal-rich stars.  The stars do not have much
    angular momentum, so also take on random orbits.  Elliptical galaxies tend
    to be more luminous towards their centers, and the luminosity drops farther
    away from the center in a smooth way.  These galaxies tend to appear redder
    in color because of their higher density of older stars and lack of new star
    formation.

    \subsubsection*{Irregular Galaxies}
    Irregular galaxies are those galaxies which do not fit into either of these
    classifications.  These galaxies vary greatly from one another, and it is
    therefore difficult to make any generalizations about them.

    Note that these classifications of galaxies do not hold up as well at higher
    redshifts, but it is believed that some of the galaxies we see farther away
    are precursors to these types of galaxies.

    \subsubsection*{}
    \vspace{-5mm}
    In order to fully understand how supermassive black holes affect a host
    galaxy throughout its lifetime, it is important to understand the major
    events that happen throughout a galaxy's lifetime.  The first major
    milestone in a galaxy's life is its formation.  This process is believed to
    be very different for spiral and elliptical galaxies.  For spiral galaxies,
    it is believed that their formation of their basic structure is caused by
    the monolithic collapse of primordial cold dust clouds.  This collapse would
    create an environment where star formation would flourish, and the intrinsic
    angular momentum of the gas would be transferred to new stars.

    On the other
    hand, it is believed that elliptical galaxies form from the merging of two
    spiral galaxies that are similar in size.  This type of merging would
    annihilate the disks, giving the resulting galaxy an elliptical shape.  If
    one galaxy was much bigger than the other, then the larger galaxy would be
    able to acquire the smaller one and still keep its general spiral
    morphology, although some of its features could change such as the size of
    the central bulge.

    These two events, formation and merging, make up the two major events in a
    galaxy's lifetime that dramatically change its morphology.  Supermassive
    black holes are believed to play significant roles in these processes, which
    will be further described below.


\subsection*{\bf Scaling Relations}
There is an incredibly tight relationship between the size of a galaxy's central
bulge and the mass of its associated supermassive black hole.

\subsection*{\bf Active Galactic Nucleus Feedback}
The supermassive black hole acts as a sort of galactic thermostat through a
process called AGN feedback.

\subsection *{\bf Galaxy Merging}
Supermassive black holes add another layer of complexity to galaxy merging, but
this extra complexity is useful in discovering why our previous computational
models differed from observational evidence.

\end{document}
