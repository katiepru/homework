\documentclass[12pt]{article}
\usepackage[margin=1in]{geometry}
\usepackage{indentfirst}

\title{Growth of Supermassive Black Holes and Scaling Relations
              Between Supermassive and Host Galaxies}
\date{}
\author{Kaitlin Poskaitis}

\begin{document}

\maketitle

Not long ago, supermassive black holes were a completely separate field of study
from that of galaxies.  However, more recent discoveries have shown that these
two types of astronomical objects are very closely related and have measurable
effects on each other.  There are numerous of these effects, and some of these
have been able to explain several of the pitfalls of our previous N-body
simulations, such as having fewer large galaxies than expected as well as
galaxies remaining hotter, and therefore less star formation occurring in older
galaxies than expected.  Therefore, one can see why it would be important to
understand the scaling relations and resulting co-evolution of supermassive
black holes and their host galaxies.

\subsection*{Black Hole Basics}
Black holes are a singularity.

\subsection*{Supermassive Black Hole Properties and Formation}
Supermassive black holes are essentially scaled-up stellar mass black holes.

\subsection*{Detecting Supermassive Black Holes}
There are different ways to detect different types of active galactic nuclei.

    \subsubsection*{Quiescent Supermassive Black Holes}
    If a supermassive black hole is truly dormant, then it can only be detected
    by its gravitational effects on surrounding visible matter.

    \subsubsection*{Quasars}
    Quasars need to be distinguished from stellar sources of radiation since
    they look very similar as they are both point sources of radiation.

    \subsubsection*{Seyfert Galaxies}

    \subsubsection*{Radio Galaxies}

\subsection*{\bf Galaxy Properties and Formation}
Galaxies can be classified in many ways.

\subsection*{\bf Scaling Relations}
There is an incredibly tight relationship between the size of a galaxy's central
bulge and the mass of its associated supermassive black hole.

\subsection*{\bf Active Galactic Nucleus Feedback}
The supermassive black hole acts as a sort of galactic thermostat through a
process called AGN feedback.

\subsection *{\bf Galaxy Merging}
Supermassive black holes add another layer of complexity to galaxy merging, but
this extra complexity is useful in discovering why our previous computational
models differed from observational evidence.

\end{document}
