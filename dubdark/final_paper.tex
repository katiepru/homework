\documentclass[12pt]{article}
\usepackage[margin=1in]{geometry}
\usepackage{indentfirst}

\title{Growth of Supermassive Black Holes and Scaling Relations
              Between Supermassive and Host Galaxies}
\date{}
\author{Kaitlin Poskaitis}

\begin{document}

\maketitle

Not long ago, supermassive black holes were a completely separate field of study
from that of galaxies.  However, more recent discoveries have shown that these
two types of astronomical objects are very closely related and have measurable
effects on each other.  There are numerous of these effects, and some of these
have been able to explain several of the pitfalls of our previous N-body
simulations, such as having fewer large galaxies than expected as well as
galaxies remaining hotter, and therefore less star formation occurring in older
galaxies than expected.  Therefore, one can see why it would be important to
understand the scaling relations and resulting co-evolution of supermassive
black holes and their host galaxies.

\subsection*{Black Hole Basics}
Black holes are a singularity.

\subsection*{Supermassive Black Hole Properties and Formation}
Supermassive black holes are essentially scaled-up stellar mass black holes.

\subsection*{Detecting Supermassive Black Holes}
There are different ways to detect different types of active galactic nuclei.

    \subsubsection*{Quiescent Supermassive Black Holes}
    If a supermassive black hole is truly dormant, then it can only be detected
    by its gravitational effects on surrounding visible matter.

    \subsubsection*{Quasars}
    Quasars need to be distinguished from stellar sources of radiation since
    they look very similar as they are both point sources of radiation.

    \subsubsection*{Seyfert Galaxies}

    \subsubsection*{Radio Galaxies}

\subsection*{\bf Galaxy Properties and Formation}
Astronomers have discovered many different galaxies throughout times, and these
galaxies can be classified in several different ways based on their different
properties.  When considering nearby galaxies $z < 1$, the Hubble
classification system works well for classifying the many different galaxies in
the Local Group and beyond (CITE hubble classifications).  Hubble's
classification system, which has been revised several times since its initial
incarnation, categorizes galaxies into three main type: spiral, elliptical, and
irregular.  These categories have several subcategories within them describing
the specific morphology of each type, but galaxies that fall into any of these
three categories have many similar characteristics.

    \subsubsection*{Spiral Galaxies}
    Spiral galaxies tend to be the go-to example for describing galaxies in general
    as the Milky Way is this type of galaxy.  Spiral galaxies have several main
    morphological components, one of which is the stellar halo (CITE Lecture 10/29).
    This is a spherical region surrounding the galaxy that is on average
    approximately 100 kpcs in size.  This is much larger than the other, more
    luminous visible components.  However, the stars that care found here are very
    sparse and tend to be old, metal-poor stars.  This is also the most common place
    for globular clusters to form.

    Another component of spiral galaxies is the central bulge, which is the very
    luminous central portion of the galaxy which forms a noticeable bulge in the
    center of the disk.  This section contains mostly old, metal-rich stars that
    follow more random orbits due to their proximity to the center.  The central
    bulge on average contains about $\frac{1}{4}$ of the mass of the entire
    galaxy, although the size of the central bulge can vary greatly between
    galaxies.  It also sometimes contains a bar of stars that is more flat.
    This bar can be very pronounced in some galaxies and seemingly nonexistent
    in others, and is one of the ways Hubble determines subcategories within the
    spiral class.

    This central bulge forms the center of the two largest structures in spiral
    galaxies: the thin and thick disks.  These disks make up most of the mass of
    the galaxy, and contain different types of stars in each.  The thin disk,
    which is much larger, contains younger, more metal-rich stars and the thick
    disk contains older, more metal-poor stars.  The stars in these disks have
    higher angular momentums and therefore do not follow random orbits like the
    ones in the central bulge.  Finally, on the edges of the disks, spiral arms
    can be observed, which contain young, metal-rich stars as well as gas and
    dust.

    As a whole, spiral galaxies tend to look very flat because of the thin and
    thick disks.  Furthermore, they tend to have a relatively high amount of new
    star formation going on because of the size and environment of the thin
    disk.  As a result, spiral galaxies tend to look bluer in color, with a more
    red center as less star formation is occurring there.

    \subsubsection*{Elliptical Galaxies}




\subsection*{\bf Scaling Relations}
There is an incredibly tight relationship between the size of a galaxy's central
bulge and the mass of its associated supermassive black hole.

\subsection*{\bf Active Galactic Nucleus Feedback}
The supermassive black hole acts as a sort of galactic thermostat through a
process called AGN feedback.

\subsection *{\bf Galaxy Merging}
Supermassive black holes add another layer of complexity to galaxy merging, but
this extra complexity is useful in discovering why our previous computational
models differed from observational evidence.

\end{document}
