\documentclass[12pt]{article}
\usepackage[margin=1.0in]{geometry}
\usepackage{indentfirst}

\begin{document}

\noindent Kaitlin Poskaitis\\
Assignment 4\\
September 27, 2013
\begin{center}
\section*{\bf The Expanding Universe and the Big Bang Theory}
\end{center}

% Intro
Once astronomers were able to understand some of the basic properties of the
universe, such as its scale and relativistic nature, they next sought to learn
more about how the universe changes state, if it did at all.  This was done
through various methods, both theoretical and observational.  In the end, it was
discovered that the universe is expanding, and that the universe began with a
massive explosion as outlined in the Big Bang Theory.  However, the process to
reach these understanding was a long one and met much opposition during the
process.  Even so, eventually observational evident became so overwhelming that
these principles are still accepted today.

% Cosmological Constant
The foundation for many of the scientific theories behind the origin and fate of
the universe came from the introduction of the Cosmological Principle.  This is
a principle which states that over a large scale, the universe is homogeneous,
meaning that it has the same density, as well as isotropic, meaning that it
looks the same in every direction (Coles 2001).  This was introduced because it
greatly simplified Einstein's 10 field equations into just one equation
relating the curvature tensor, the metric, and the cosmological constant
(Lecture 9/24).  This is because if the density of the universe is consider to
be the same, then that means that the overall curvature of the universe is also
the same, even though there are blips here and there from the places with
matter. This principle is believed to be true because the universe is
mostly empty space, so over a large enough area the density would be very small
and mostly uniform.

% Einstein, Friedmann, and Le Maitre models
Once the Cosmological Principle was accepted, several views of the universe were
formed by prominent astronomers of the time.  Einstein, for example, modelled
the universe after his field equations and his theories of relativity (Center
for history of Physics 2013).  However,
he found that his model could not contain any matter in it and still model a
static universe as the gravitational forces would cause the universe to promptly
collapse on itself, which is what most astronomers believed was the case.  In
order
to combat this problem, he found that having a nonzero cosmological constant in
his field equations would yield a static universe even if it contained matter.
He believed it only to be a hypothetical term and not required by the theory
(Center for History of Physics 2013).

Even though Einstein's model seemed to fit his equations within a static
universe, a Russian astronomer named Alexander Friedmann came up with a
different model that fit in with Einstein's field equation without the need for
a nonzero cosmological constant (Coles 2001).  In order for these conditions to
be met, Friedmann determined that the universe needed to be expanding.
Furthermore, he concluded that a higher density of matter implies an earlier
time and vice versa.  The outcome of this is that, since the universe is
homogeneous, overall curvature of the universe under this model only has three
options: flat, positively curved (like a four-dimensional sphere),
and negatively curved (like a four-dimensional saddle) (Coles 2001).  The fate
of the universe is directly related to which of these geometries it conforms to.
If the universe is open or negatively curved, then it can expand forever;
however, if it is positively curved, then eventually all of the matter will pull
back together in a big crunch (Coles 2001).

Friedmann's model was a very innovative approach to analyzing Einstein's
equation, but it was a very obscure Belgian astrophysicist that was able to
develop this theory further, even though he independently cam up with it (Center
for history of Physics 2013).  His
model also described an expanding universe and shared many of the conclusions of
Friedmann's model.  However, he was also able to speculate on what this model
meant for the beginning of the universe, and he theorized that the universe
began in a great but short explosion.  This was the first version of the Big
Bang Theory of the origin of the universe (Center for History of Physics 2013).

% Huble's stuff
Even though there had been several theories about an expanding universe, the
concept of such an idea seemed ridiculous at the time.  Everyone had assumed
that the universe was static and that we were not at a special time in its
history, as there had been no observational evidence that supported otherwise
(Center for History of Physics 2013).
However, Edwin Hubble discovered an interesting effect that solidified the
concept of an expanding universe: he discovered that there was a linear
relationship between the distance a galaxy is from us and the velocity at which
it is moving away from us.  Assuming that we are not in a special place in the
universe, this strongly suggests that the entire universe is expanding (Center
for History of Physics 2013).

Hubble's
discovery also led to the idea of the Big Bang because if the universe has been
expanding, then that means that it was much denser before.  Following this logic
all the way, this means that at one point everything was at one point and
exploded to begin the expansion of the universe (Center for History of Physics
2013).  However, many astronomers did not believe this at first as the data
behind this showed the universe to be much younger than previous data has shown.
As a result, another theory was created by an astronomer named Hoyle, termed the
Steady State Theory.  This theory suggested that as the universe continued to
expand, more matter was spontaneously created to keep the density constant.
This would allow for an expanding universe without the need for a singularity in
the beginning, and the age of the universe could fit with previous data.  The
main argument for this theory was that if we are not at a special place in
space, then we should not be at a special point in time either.  However, this
was eventually defeated as amendments to the data for the Big Bang concluded
that the age of the universe did in fact line up with previous data.
Furthermore, a background of microwave radiation was found to be consistent
throughout the observable universe, which also lines up with the Big Bang
Theory (Center for History of Physics 2013).

In all, the discovery of the expanding universe and the Big Bang Theory of the
origin of the universe was a long and uphill battle to determine how to
interpret Einstein's equations.  This was finally aided by Hubble's discovery of
the velocity to distance proportion, which created a concrete foundation for an
expanding universe and a Big Bang origin.  Even though the Steady State Theory
seemed to be more natural to many, the discovery of the cosmic microwave
background finally deemed it to be implausible, and thus leading us to our
near current understanding of these properties of the universe today.


\end{document}
