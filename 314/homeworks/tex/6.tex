\documentclass[11pt]{article}
\usepackage[margin=0.75in]{geometry}
\usepackage{sectsty}
\usepackage{indentfirst}
\sectionfont{\large}

\title{\bf Kaitlin Poskaitis - Homework 6}
\author{Section 2}
\date{}

\begin{document}

\maketitle

\section{Parameter Passing}
\begin{enumerate}
	\item RISC Machine Instructions \\
	LOADI r1 \#4\\
	ADD r2 r1 r0\\
	LOADI r3 \#5\\
	STORE r2 r3\\

	LOADI r4 \#-12\\
	ADD r5 r4 r0\\
	LOAD r6 r5\\
	LOAD r7 r6\\
	
	LOADI r8 \#-8\\
	ADD r9 r8 r0\\
	LOAD r10 r9\\

	ADD r11 r7 r10\\
	LOAD r12 r2\\
	ADD r13 r11 r12\\
	STORE r6 r13\\

	LOADI r14 \#1\\
	STORE r9 r14

	\item What values for a and b does that program print?\\
	a = 8\\
	b = 2


\end{enumerate}

\section{Scheme}

\begin{enumerate}
	\item 
	(cons 'a (cons 'b 
		(cons (cons 'c (cons 'd 
			(cons (cons 'e (cons 'f 
				(cons (cons 'g '()) '())
			)) '())
		)) '()) 
	))
	\item
	(cons
		(cons 
			(cons 
				(cons 'a '())
			(cons 'b (cons 'c '())))
		(cons 'd 
			(cons (cons 'e (cons 'f '())) '())	
		))
	(cons 'g '()))
	\item
	(cons 'a (cons + (cons 'b '())))

\end{enumerate}

\section{Scheme}

\begin{enumerate}
	\item
	(define flatten
	 (lambda (l)
	  (cond 
	   ((null? l)
		empty)
	   ((list? (car l))
		(cons (flatten (car l)) (flatten (cdr l))))
	   (else
		(cons (car l) (flatten (cdr l)))))))
	\item 


\end{enumerate}

\end{document}
