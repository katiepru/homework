\documentclass[12pt]{article}
\usepackage[margin=1.0in]{geometry}
\usepackage{sectsty}
\usepackage{indentfirst}
\usepackage{listings}
\usepackage{csvsimple}
\usepackage{longtable}
\sectionfont{\large}

\title{\bf Kaitlin Poskaitis - Project 1 Report}
\author{}
\date{}

\begin{document}

\maketitle

\section{Algorithms Used}

    \subsection{Naive Top Down Allocator}
    The naive top down allocator uses a fairly simple algorithm to do the
    register allocation. The heuristic it uses is the number of times each
    register is used. This is calculated for each virtual register when the file
    is initially parsed. Simply put, it sorts the list of virtual registers used
    in the program based on the number of times they are used in the program,
    and allocates the most frequently used ones to the physical registers.
    Everything else gets spilled. If there are ties, the winner is arbitrarily
    chosen based on the particular order of the sorted list. All virtual
    registers that are not allocated physical registers are spilled to an
    address in the form $r0 - 4c$ where $r0$ contains 1020 and $c$ is between 0
    and 254. All physical register assignments are permanent, meaning that
    if a virtual register is assigned to physical register r1, another virtual
    register will never be assigned to r1.

    Once the register assignments are computed, assign the rest of the virtual
    register a spill address. Using these mappings, the code can be generated.
    This is done by going through the list of instructions line by line,
    checking if we need to load any of the input virtual registers (if they are
    spilled), change the virtual registers to the physical registers that
    contain their values, and spill the result if it needs to be.

    \subsection{Top Down Allocator Based on Live Ranges}
    The more complex version of the simple top down allocator uses the notion of
    live ranges to (hopefully) more efficiently allocate and assign physical
    registers to the program. The live range of a virtual register is defined as
    the range of line numbers in the original program such that the register has
    been written to and will be used again. For example, if r1 is written to on
    line 3, used once in line 5, and used for the last time in line 7, its live
    range would be 3-6. These live ranges are calculated when the program file
    is initially parsed.

    The actual allocation leverages these live ranges to know when it is safe to
    reuse registers. It follows a similar set of rules from the simple version
    of the algorithm. Specifically, once a virtual register is assigned a
    physical register, it stays in that register for the entirety of its live
    range. Also, if a register is spilled, it will never be assigned a physical
    register. Any physical register that it uses will be part of the feasible
    set.

    The actual allocation algorithm goes as follows: first, determine which
    registers are live at each line in the program and put these in a list. Sort
    the resulting list of lists in descending order by the length of the sub
    lists. Then, for each sub list in the main list, determine how many virtual
    registers are not spilled. If that number is greater than the number of
    allocatable registers, spill as many registers as needed to get these
    numbers to equal. In order to decide which registers to spill, spill the
    ones with the fewest number of occurrences first. If there are ties, spill
    the one with the longer live range, as this is likely to resolve other spill
    conflicts farther along. Once the number of live and not spilled registers
    on that line is less than or equal to the number of allocatable registers,
    move on to the next sub list. Note that this does not decide which virtual
    registers to allocate physical registers, rather which virtual registers to
    not allocate physical registers to. This can result in some lines not fully
    utilizing all allocatable registers, but does guarantee if something is
    spilled, it is not used more than something that is allocated in an
    overlapping live range.

    Once it is decided which registers to spill, it is then possible to do the
    register assignment. In order to do this, I go through the instructions line
    by line and see which virtual registers are not yet allocated a physical
    register that should be. Check the list of physical registers for one that
    is available, meaning either it is unassigned or the virtual register it was
    assigned to is outside of its live range, and use one that fits this
    criteria.

    Once the spill addresses and register assignments are computed, use this
    mapping to generate the code as per the simple top down allocator.


    \subsection{Bottom Up Allocator}
    The bottom up allocator works very differently from either of the top down
    allocators, but is in itself not a difficult algorithm. Essentially, the
    idea behind this algorithm is to have all virtual registers for a
    particular instruction loaded into an allocatable register. In order to make
    room if they are all full, the virtual register with the farthest next use
    will be spilled to the next available spill address (in the same form as the
    top down allocators.

    To detail the algorithm further, first ensure that all source virtual
    registers are loaded into physical registers. If not, they need to be loaded
    into a register from their spill locations. If there are available
    registers, meaning that they have not been assigned or are currently
    assigned to a virtual register that has exited its live range. If there are
    not any available physical registers by this metric, then one needs to get
    spilled for each source virtual register that is not yet loaded. In order to
    do this, the program iterates over all physical allocatable registers and
    determines the one containing the virtual register with the farthest next
    use. The lines that the virtual registers are used on are in a list called
    VRegister.uses. Once we have ensured that all sources are loaded, we need to
    find an allocatable physical register to put the written result into, if
    there is one. This is done by the same process as is discussed above, but we
    are able to reuse physical registers used by this instruction's sources if
    needed. Finally, the virtual registers in the line are replaced by the
    physical registers that contain those values.



\section{Performance Testing Overview}

    \subsection{Methods Used}
    In order to test the performance of each algorithm, several metrics will be
    used. First, in order to ensure that the allocator itself does not add too
    much overhead to the compilation process, its execution will be times for
    each of the report blocks with 5, 10, and 20 registers. Since register
    allocation and assignment is an NP complete problem, this time is used to
    make sure that the heuristics used do not add too much complexity such that
    actually using them is infeasible.

    Next, in order to test the performance of the generated code, both
    instruction count and cycle count will be used. Instruction count will refer
    to to total number of instructions added by the allocator, which can give an
    approximate idea of how often registers were spilled and/or loaded. However,
    about half of these instructions will be much more costly than the other
    half, so to get a better measure of code performance, the best way is to run
    it and count the cycles. The load and store operations will add the most
    cycles, as accessing memory is much slower than accessing registers.
    Checking which algorithm generally results in fewer additional cycles will
    show which algorithm is better in most cases.

    \subsection{Hypotheses}
    I predict that the complex top down algorithm will almost always outperform
    the simple top down algorithm because taking live ranges into consideration
    will allow for register reuse. I also predict that the performance of the
    complex top down relative to the bottom up algorithm will vary greatly based
    on the program, but that they will be generally similar. This is because the
    live range top down will not perform as well when many virtual registers
    have very long live ranges, resulting in a similar scenario to the naive
    algorithm.

    \newpage


\section{Performance Test Data}

    \subsection*{Table 1: Allocator Execution Performance}
    \begin{longtable}{|l|c|c|c|c|c|}
        \bfseries Block & \bfseries Algo & \bfseries Alloc Time
        \csvreader[head to column names]{times.csv}{}
        {\\\hline\block & \algo & \alloctime}
    \end{longtable}



\section{Performance Test Analysis}


\end{document}
