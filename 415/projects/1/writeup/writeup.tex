\documentclass[12pt]{article}
\usepackage[margin=1.0in]{geometry}
\usepackage{sectsty}
\usepackage{indentfirst}
\usepackage{qtree}
\usepackage{listings}
\usepackage{csvsimple}
\sectionfont{\large}

\title{\bf Kaitlin Poskaitis - Project 1 Report}
\author{}
\date{}

\begin{document}

\maketitle

\section*{Algorithms Used}

    \subsection{Naive Top Down Allocator}
    The naive top down allocator uses a fairly simple algorithm to do the
    register allocation. The heuristic it uses is the number of times each
    register is used. This is calculated for each virtual register when the file
    is initially parsed. Simply put, it sorts the list of virtual registers used
    in the program based on the number of times they are used in the program,
    and allocates the most frequently used ones to the physical registers.
    Everything else gets spilled. If there are ties, the winner is arbitrarily
    chosen based on the particular order of the sorted list. All virtual
    registers that are not allocated physical registers are spilled to an
    address in the form $r0 - 4c$ where $r0$ contains 1020 and $c$ is between 0
    and 254. All physical register assignments are permanent, meaning that
    if a virtual register is assigned to physical register r1, another virtual
    register will never be assigned to r1.

    \subsection{Top Down Allocator Based on Live Ranges}
    The more complex version of the simple top down allocator uses the notion of
    live ranges to (hopefully) more efficiently allocate and assign physical
    registers to the program. The live range of a virtual register is defined as
    the range of line numbers in the original program such that the register has
    been written to and will be used again. For example, if r1 is written to on
    line 3, used once in line 5, and used for the last time in line 7, its live
    range would be 3-6. These live ranges are calculated when the program file
    is initially parsed.

    The actual allocation leverages these live ranges to know when it is safe to
    reuse registers. It follows a similar set of rules from the simple version
    of the algorithm. Specifically, once a virtual register is assigned a
    physical register, it stays in that register for the entirety of its live
    range. Also, if a register is spilled, it will never be assigned a physical
    register. Any physical register that it uses will be part of the feasible
    set.

    The actual allocation algorithm goes as follows: first, determine which
    registers are live at each line in the program and put these in a list. Sort
    the resulting list of lists in descending order by the length of the sub
    lists. Then, for each sub list in the main list, determine how many virtual
    registers are not spilled. If that number is greater than the number of
    allocatable registers, spill as many registers as needed to get these
    numbers to equal. In order to decide which registers to spill, spill the
    ones with the fewest number of occurrences first. If there are ties, spill
    the one with the longer live range, as this is likely to resolve other spill
    conflicts farther along. Once the number of live and not spilled registers
    on that line is less than or equal to the number of allocatable registers,
    move on to the next sub list. Note that this does not decide which virtual
    registers to allocate physical registers, rather which virtual registers to
    not allocate physical registers to. This can result in some lines not fully
    utilizing all allocatable registers, but does guarantee if something is
    spilled, it is not used more than something that is allocated in an
    overlapping live range.

    Once it is decided which registers to spill, it is then possible to do the
    register assignment.


    \subsection{Bottom Up Allocator}



\section*{Performance Testing Overview}



\section*{Performance Test Data}

    \subsection*{Table 1: Allocator Execution Performance}
    \begin{tabular}{|l|c|c|c|c|c|}
        \bfseries Block & \bfseries Algo & \bfseries Alloc Time
        \csvreader[head to column names]{times.csv}{}
        {\\\hline\block & \algo & \alloctime}
    \end{tabular}



\section*{Performance Test Analysis}


\end{document}
