\documentclass[11pt]{article}
\usepackage[margin=0.75in]{geometry}
\usepackage{sectsty}
\usepackage{indentfirst}
\usepackage{qtree}
\usepackage{listings}
\sectionfont{\large}

\title{\bf Kaitlin Poskaitis - Homework 1}
\author{}
\date{}

\begin{document}

\maketitle

\section{ILOC Programming}

Register-register model:
\lstinputlisting{prob1_reg.i}
This resulted in 22 instructions and 26 cycles.

\vspace{5mm}

Memory-memory model:
\lstinputlisting{prob1_mem.i}
This resulted in 60 instructions and 171 cycles.

\section{Feasible Registers for ILOC}
The number of physical registers that need to be in the feasible set for ILOC is
2.  This is because the largest number of register sources that any instruction
has is 2. We are able to reuse one of these 2 registers in the feasible set.

\noindent Example is f1 and f2 are the feasible set:\\
mult f1, f2 =$>$ f1

\section{Top Down Register Allocation}
\subsection{}
\lstinputlisting{prob3_virt.i}

\vspace{5mm}
\subsection{}
\noindent r1:  2-10\\
r2:  3-4\\
r3:  4-6\\
r4:  5-8\\
r5:  6-8\\
r6:  7-7\\
r7:  8-9\\
r8:  9-9\\
r9:  10-10\\
r10: 11-11\\
MAX\_LIVE is 4

\subsection{}
\subsubsection{}
We have 3 registers total, with 2 in the feasible set.
The heuristic is to spill registers with the fewest uses until we only need 1.

\noindent Total uses:\\
r1:  2\\
r2:  2\\
r3:  1\\
r4:  2\\
r5:  3\\
r6:  1\\
r7:  1\\
r8:  1\\
r9:  1\\
r10: 1\\

\noindent r3 and r4 will be feasible set.
r5 (renamed to r1) will not be spilled.

\lstinputlisting{prob3_nomax.i}

\subsubsection{}
We again have 3 registers total, with 1 in the feasible set.
We need to consider the live ranges of each virtual register.
They are as followed:

\noindent1.\\
2.  r1\\
3.  r1,r2\\
4.  r1,r2,r3\\
5.  r1,r3,r4\\
6.  r1,r3,r4,r5\\
7.  r1,r4,r5,r6\\
8.  r1,r4,r5,r7\\
9.  r1,r7,r8\\
10. r1,r9\\
11. r10\\

\noindent We need to spill some registers such that no line exceeds 2 active registers.
To start, the lines with the largest amount of live registers will be considered
first.  The registers with the least number of uses will be spilled, and ties
will be resolved by picking the register with the largest live range.

\noindent Starting at line 7, r4, r6, and r1 will be spilled.\\
r7 will be spilled to cover line 8.\\
r3 will be spilled to cover the rest.\\
This is enough to only use 2 active registers.

\lstinputlisting{prob3_live.i}


\end{document}
