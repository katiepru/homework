\documentclass[11pt]{article}
\usepackage[margin=1.0in]{geometry}
\usepackage{indentfirst}
\usepackage{sectsty}
\sectionfont{\large}

\title{\bf Assignment 3 - Formula}
\author{Kaitlin Poskaitis}
\date{}

\begin{document}

\maketitle

Note that I used 64 bit assembly for this project (mystery\_new.s).

\section*{Figuring it Out}

I figured out this program by tracing the assembly line by line and keeping 
track of what was in each register and memory location. I first noted that it
was fibonacci because of the pattern in the output as well as the function name
in the file. I then deduced that it was recursive by noting that compute\_fib 
was called from within itself.  I then was able to figure out that the 
optimization was caching values in an array by tracing the main function and 
seeing that it allocated an array of size 47. This was also the cut off int
for the input, so caching made sense.

\section*{Optimization}

It seemed the compiler optimization removed unnecessary register and memory 
movement and storage. It seemed to condense the generated assembly considerably
by cutting out unnecessary steps with values between registers.

\end{document}
