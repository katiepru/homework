\documentclass[11pt]{article}
\usepackage[margin=1.0in]{geometry}
\usepackage{indentfirst}
\usepackage{sectsty}
\sectionfont{\large}

\title{\bf Assignment 1 - Wordstat}
\author{Kaitlin Poskaitis}
\date{}

\begin{document}

\maketitle

\section*{Algorithm}

This project works only for integers up to 32 bits. It works by converting all 
of the inputs into long ints through their appropriate conversions. It then does
the arithmetic on those long ints, and procedes to convert them into their 
specified output format through the appropriate conversion. 

\section*{Big O Analysis}

Time complexity: O(n)

The time complexity of this algorith is O(n). The algorithm can be split into 
three main parts: converting to a long int, doing the operation, and converting 
to the output format. The first part is O(n) as we have to loop over the string
to do the conversion. The second part is O(1), as it is a simple arithmetic 
operation. Lastly, the last part is at worst O(n) because the int must be 
converted to a string.

Space complexity: O(n)

The space complexity is also O(n) because 

\section*{Difficulties}

The biggest challenge of this assignment was coming up with an efficient
algorithm, as linear time was not possible with the basic structures we learned
in Data Structures. Once i was able to come up with this algorithm, the 
implementation was pretty straightforward. The most difficult concept for me to
get used to was how C handles strings as they are so different from Java. Once
I got this worked out, it was not terribly difficult.

\end{document}
