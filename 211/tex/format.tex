\documentclass[11pt]{article}
\usepackage[margin=1.0in]{geometry}
\usepackage{indentfirst}
\usepackage{sectsty}
\sectionfont{\large}

\title{\bf Assignment 1 - Wordstat}
\author{Kaitlin Poskaitis}
\date{}

\begin{document}

\maketitle

\section*{Algorithm}

The algorithm for this program is pretty straightforward. If the number is
an int, then I convert from twos complement to decimal, and then use the 
int\_to\_ascii function to convert each digit into a character in order to 
create a string. If the number is a float, I used to iee single precision 
convserion mathmatical formula in order to do the conversion. This is then 
printed in scientific notation.

\section*{Big O Analysis}

Time complexity: O(n)

This algorithm is O(n). Converting from twos complement or ieee single
precision floating points is order n as it requires one pass over the string.
Then, printing each type accordingly is also O(n).

Space complexity: O(n)

This algorithm is also O(n) space complexity as each string number is converted 
to a long int or float, which is O(n).


\section*{Difficulties}

The biggest difficulty was understanding the algorithm and implementing it 
correctly for all different types of cases.

\end{document}
