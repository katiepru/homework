\documentclass[12pt]{article}
\usepackage[margin=0.75in]{geometry}
\usepackage{indentfirst}
\usepackage{tikz}
\usepackage{pgfplots}

\title{\bf Money and Banking - Assignment 2}
\author{Kaitlin Poskaitis, Joshua Matthews, Natalie Portman}
\date{}

\begin{document}

\maketitle

\section{C = \$400 billion. D = \$1200 billion. r = R/D = 0.1. e=0}

\begin{enumerate}
	\item {\bf Calculate the monetary base B}
	First, we need to calculate R. R = rD = \$120 billion.\\
	$B = C + R. B = 400 + 120. B = \$520$ billion. 
	\item {\bf Calculate the money mulitplier, m}
	$c = \frac{C}{D}. c = \frac{400}{1200} = \frac{1}{3}$
	$m = \frac{1+c}{r+e+c} = \frac{\frac{4}{3}}{\frac{13}{30}} = 
	\frac{40}{13} \approx 3.077$
	\item {\bf What is the money supply?}
	$M = 520*\frac{40}{13} \approx  \$1600 billion$
\end{enumerate}

\section{Money supply and its determinants}

\begin{tabular}{|l |c| r|}
	\hline
	& August 1929 & March 1933 \\ \hline
	Reserve-deposit ratio (r) & 0.14 & 0.21 \\ \hline
	Currency-deposit ratio (c) & 0.17 & 0.41 \\ \hline
	Excess-reserve ration (e) & 0.12 & 0.12 \\ \hline
\end{tabular}

\begin{enumerate}
	\item Calculate the money mulitplier in August 1929.\\
	$m = \frac{1+c}{r+e+c} = \frac{1.17}{0.43} \approx 2.721$
	\item Calculate the money multiplier in March 1933. \\
	$m = \frac{1+c}{r+e+c} = \frac{1.41}{0.74} \approx 1.905$
	\item If the monetary base is the same in both years, in which
	year was the money supply higher?\\
	The oney supply was higher in 1929 because the money multiplier is higher, 
	and the money supply is the product of the monetary base and the money 
	mulitplier.
	\item Currency-deposit ratio rises from 1929 to 1933. Suppose the 
	Reserve-deposit ratio remains unchanged at the 1929 level. Calculate the 
	money multiplier in August 1933. What would happen to money supply 
	(increase / decrease) in 1933 compared to 1929? e is unchanged.

	\item Reserve-deposit ratio rises from 1929 to 1933. Suppose the 
	currency-deposit ratio remains unchanged at the 1929 level. Calculate the 
	money multiplier in August 1933. What would happen to money supply (increase
	/ decrease) in 1933 compared to 1929? e is unchanged\\
	$c = 0.17. m = \frac{1+c}{r+e+c} = \frac{1.17}{0.5} \approx 2.340$\\
	The money supply would decrease relative to that of 1929, even though the 
	money multiplier in 1933 is greater than it was with the original data.

\end{enumerate}

\section{If the Fed sells \$2 million of bonds to Irving the investor, who pays 
for the bonds with a briefcase filled with currency, what happens to the 
monetary base? Explain in details using the T-accounts.}

\section{If the Fed lends banks an additional total of \$100 million but 
depositors withdraw \$50 million and hold it as currency, what happens to 
reserves and the monetary base? Use T- accounts to explain your answer.}

\section{Using T- accounts, show what happens to checkable deposits in the 
banking system when the Fed lends an additional \$1 million to the First 
National Bank.}

\section{How does a general increase in uncertainty as a result of a failure of 
a major financial institution lead to an increase in adverse selection and moral
hazard problems?}

\section{What happens to reserves at the First National Bank if one person 
withdraws \$1,000 of cash and another per-son deposits \$500 of cash? Use T- 
accounts to explain your answer.}

\section{Conduct a gap analysis for the bank, that has \$15 million of fixed- 
rate assets, \$30 million of rate- sensitive assets, \$25 million of fixed- rate
liabilities, and \$20 million of rate- sensitive liabilities. show what will 
happen to bank profits if interest rates rise by 5 percentage points. 
What actions could you take to reduce the bank’s interest- rate risk?}








\end{document}
