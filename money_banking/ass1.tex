\documentclass[11pt]{article}
\usepackage[margin=0.75in]{geometry}
\usepackage{indentfirst}

\title{\bf Money and Banking - Assignment 1}
\author{Kaitlin Poskaitis and Joshua Matthews}
\date{}

\begin{document}

\maketitle

\section{What is a bond.s yield to maturity if it has a \$1000 face value and a
coupon rate of 10? The bond is currently selling for \$1,044.98 and has 2 years
to maturity.}

\section{What is the yield to maturity on a \$1,000-face value discount bond
maturing in one year that sells for \$800?}

\section{Explain why would you be more or less willing to buy an Apple bond in
the following situations:}

{\bf a.} Your wealth increases.

{\bf b.} The US Treasury bonds become more liquid.

{\bf c.} Expect gold to appreciate in value.

\section{Using the supply and demand for bonds framework show what the effect is
on interest rates when the riskiness of bonds rises. Use appropriate supply and
demand diagrams.}

\section{Suppose there are 2 kinds of bonds- default free Treasury bonds and
corporate bonds. Risk premium on corporate bonds are usually anticyclical; 
that is, they decrease during business cycle expansions and increase during
recessions. Why is this so?}

\section{Suppose that income tax rates on the return on US Treasury bonds are
reduced such that the latter's after-tax expected return relative to municipal
bonds become greater. What effect would this have on the price and hence 
interest rates of municipal bonds?}

\section{Compute the price of a stock that pays \$1 per year dividend and that
you expect to be able to sell in one year for \$20, assuming you require a 
15\% return.}

\section{If a corporation is expected to lose \$5 per share this year and it
actually loses \$4, what does the efficient market hypothesis say will happen 
to the price of stock when the actual loss is announced?}

\end{document}
