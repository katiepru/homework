\documentclass[12pt]{article}
\usepackage[margin=0.75in]{geometry}
\usepackage{indentfirst}
\usepackage{tikz}
\usepackage{pgfplots}

\title{\bf Money and Banking - Assignment 1}
\author{Kaitlin Poskaitis and Joshua Matthews}
\date{}

\begin{document}

\maketitle

%Question 1
\section{What is a bond's yield to maturity if it has a \$1000 face value and a
coupon rate of 10\%? The bond is currently selling for \$1,044.98 and has 2 
years to maturity.}

\indent $P = \frac{C}{1+i} + \frac{C}{(1+i)^2}$

Where C = Coupon payment, F = Face value, P = Price, and i = Yield to maturity

\indent $1044.98 = \frac{100}{1 + i} + \frac{100}{(1 + i)^2}$

\indent $i = 7.495\%$


%Question 2
\section{What is the yield to maturity on a \$1,000-face value discount bond
maturing in one year that sells for \$800?}

\indent $i = \frac{F - P}{P}$

Where F = Face value, i = Yield to maturity, and P = Current price

\indent $i = \frac{200}{800}$

\indent $i = 25\%$


%Question 3
\section{Explain why would you be more or less willing to buy an Apple bond in
the following situations:}

{\bf a.} Your wealth increases.

You would be more likely to buy an Apple bond because when your wealth 
increases, you will want to purchase more assets. A bond would fall under this 
category, so you will likely be more willing to buy an Apple bond.

{\bf b.} The US Treasury bonds become more liquid.

You would be less likely to buy an Apple bond in this case because the Apple
bond becomes less liquid than the US Treasury bonds, and the demand for an 
asset is positively related to its liquidity relative to other assets, so
you would be less willing to buy an Apple bond as a result.

{\bf c.} Expect gold to appreciate in value.

You would be less likely to buy an Apple bond because, assuming gold is a 
substitute for the Apple bond, the demand for gold would increase and, as a
result, the demand for the Apple bond will decrease since gold is anticipated
to appreciate in value.

%Question 4
\section{Using the supply and demand for bonds framework show what the effect is
on interest rates when the riskiness of bonds rises. Use appropriate supply and
demand diagrams.}

\begin{tikzpicture}
	\begin{axis}[domain=0:10]
	\addplot[mark=none, samples=100, red] function{x};
	\addplot[mark=none, samples=100, blue] function {-x};
	\end{axis}
\end{tikzpicture}

%Question 5
\section{Suppose there are 2 kinds of bonds- default free Treasury bonds and
corporate bonds. Risk premium on corporate bonds are usually anticyclical; 
that is, they decrease during business cycle expansions and increase during
recessions. Why is this so?}

%Question 6
\section{Suppose that income tax rates on the return on US Treasury bonds are
reduced such that the latter's after-tax expected return relative to municipal
bonds become greater. What effect would this have on the price and hence 
interest rates of municipal bonds?}

%Question 7
\section{Compute the price of a stock that pays \$1 per year dividend and that
you expect to be able to sell in one year for \$20, assuming you require a 
15\% return.}

\indent $P = \frac{D + P1}{1 + R}$

Where P = Price of stock, D = Dividend, P1 = Expected price, and R = Expected 
return.

\indent $P = \frac{1 + 20}{1 + 0.15}$ \indent $P = \frac{21}{1.15}$ 
\indent $P = \$18.26$

%Question 8
\section{If a corporation is expected to lose \$5 per share this year and it
actually loses \$4, what does the efficient market hypothesis say will happen 
to the price of stock when the actual loss is announced?}

\end{document}
