\documentclass[11pt]{article}
\usepackage{fullpage}
\usepackage{amsthm}
\usepackage{amsthm,amsmath,amsfonts,amssymb,amstext}
\usepackage{latexsym,ifthen,url,rotating}
\usepackage[usenames,dvipsnames]{color}


% --- -----------------------------------------------------------------
% --- Document-specific definitions.
% --- -----------------------------------------------------------------
\newtheorem{definition}{Definition}

\newcommand{\concat}{{\,\|\,}}
\newcommand{\bits}{\{0,1\}}
\newcommand{\Range}{{\mathrm{Range}}}
\newcommand{\A}{{\mathcal{A}}}
\renewcommand{\arraystretch}{1.5}

% --- -----------------------------------------------------------------
% --- The document starts here.
% --- -----------------------------------------------------------------
\begin{document}
%\maketitle
\sloppy

\noindent Kaitlin Poskaitis\\
CS206: Introduction to Discrete Structures II, Spring 2013\\
Section 1\\

\begin{center}
\LARGE{\textbf{Homework 4}}\\
\large{\textbf{\emph{Due at the beginning of class on Wednesday, March 13}}}
\end{center}

\vspace{.1in}

\noindent\textbf{Instructions:} Point values for each problem are listed.
Write your solutions neatly or type them up.  Typed solutions will also be
accepted via Sakai.

\begin{enumerate}

\item (3 points) In the card game bridge we deal 13-card hands to 4 players
named North, South, East, and West (all 52 cards are dealt).  What is the
probability that East and West have no spades?

Number of ways to deal them no spades: $\binom{39}{26}$

Number of ways to deal them hands: $\binom{52}{26}$

Probability of neither of them getting a spade: 
$\frac{\binom{39}{26}}{\binom{52}{26}}$

\item (4 points) An urn contains $n>0$ white balls and $m>0$ black balls.
Suppose we draw two balls without replacement.  What is the probability that
the balls are of the same color?  What if we draw them with replacement?  Show
you work.  Which of these probabilities is larger?  Briefly explain some
intuition for why one should be larger.

{\bf No replacement:}

	Let A be the event that the second ball is white. \\
	Let B be the event that the first ball is white. \\
	P(B) = $\frac{n}{n+m}$ \\
	P(A$|$B) = $\frac{n-1}{n+m-1}$ \\
	Need $P(A \cap B)$ \\
	$P(A \cap B)$ = P(A$|$B) * P(B) \\
	$P(A \cap B)$ = $\frac{n}{n+m} * \frac{n-1}{n+m-1}$ \\
	Let C be the event that the second ball is black. \\
	Let D be the event that the first ball is black. \\
	P(D) = $\frac{m}{n+m}$ \\
	P(C$|$D) = $\frac{m-1}{n+m-1}$ \\
	Need $P(C \cap D)$ \\
	$P(C \cap D)$ = P(C$|$D) * P(D) \\
	$P(C \cap D)$ = $\frac{m}{n+m} * \frac{m-1}{n+m-1}$ \\
	Probability that both balls drawn are the same color is: \\
	$P(A \cap B) + P(C \cap D)$
	$(\frac{n}{n+m} * \frac{n-1}{n+m-1}) + 
	(\frac{m}{n+m} * \frac{m-1}{n+m-1})$

{\bf With replacement:}

	The probability that 2 white balls were picked: $(\frac{n}{n+m})^{2}$ \\
	The probability that 2 black balls were picked: $(\frac{m}{n+m})^{2}$ \\
	Therefore, the probability that both drawn were the same color is: \\
	$(\frac{n}{n+m})^{2}$ + $(\frac{m}{n+m})^{2}$ 

{\bf Which one is larger}

	Intuitively, the probability of picking 2 balls of the same color with
	replcement should be greater because there will be a greater chance of 
	picking the second ball as the same color as the first with replacement.

\item (4 points) Again consider an urn with $n>0$ white balls and $m >0$ black
balls.  Suppose we draw $r\geq 1$ balls from the urn without replacement.  What
is the probability that we draw exactly $k$ white balls?

Assuming $r\geq n$ and $r\geq m$, number of different combinations is r+1. \\
Number of ways to get eactly k whites: 1. \\
Probability og getting exactly k whites: $\frac{1}{r+1}$

\item (4 points) A box contains a mixture of cubes and spheres and any of these
objects can be either white or black.  Suppose the box contains $4$ black
cubes, $6$ black spheres, $6$ white cubes, and $x$ white spheres.  Consider the
experiment of drawing a random object from the box, and let $A$ be the event
that a cube is drawn and $B$ be the event that a black object is drawn.
If $A$ and $B$ are independent, what is $x$?

	Let A be the event that a cube was drawn. \\
	Let B be the event that a black object was drawn. \\
	If these events are independent, then $P(A \cap B) = P(A) * P(B)$ \\
	$P(A) = \frac{10}{16+x}$ \\
	$P(B) = \frac{10}{16+x}$ \\
	$P(A \cap B) = \frac{4}{16+x}$ \\
	$\frac{4}{16+x} = \frac{10}{16+x} * \frac{10}{16+x}$ \\
	$\frac{4}{16+x} = \frac{100}{(16+x)^{2}}$ \\
	$\frac{4}{16+x} = 100$ \\
	$x = 9$ white spheres


\item (10 points total) Two fair dice are rolled.  Define the random variables
$X =$ the sum of the two rolls,
$Y =$ the maximum of the two rolls,
$Z =$ the absolute value of the difference of the two rolls
and $W = XY$ (i.e., the product of $X$ and $Y$).
\begin{enumerate}
 \item (2 points) What are $\Range(X)$, $\Range(Y)$, $\Range(Z)$ and
 $\Range(W)$?

 \item (2 points) What are the partitions $\A_X$ and $\A_Z$?
 \item (3 points) Give tables showing the values of $f_X,f_Y,f_Z,$ and $f_W$.
 \item (3 points) Are the events $X =7$ and $Z=1$ independent?
\end{enumerate}

\begin{enumerate}
	\item $\Range(X)$ = [2, 12], $\Range(Y)$ = [1, 6], $\Range(Z)$ = [0, 5]
	$\Range(W)$ = [2, 72]
	\item $\A_X = \{\{(1,1)\}, \{(2,1),(1,2)\}, \{(2,2),(3,1),(1,3)\}, 
	\{(3,2),(2,3),(4,1),(1,4)\}, $\\$\{(5,1),(1,5),(2,4),(4,2),(3,3)\}, 
	$\\$\{(6,1),(1,6),(2,5),(5,2)(3,4),(4,3)\}, $\\$
	\{(4,4),(3,5),(5,3),(6,2),(2,6)\}, $\\$
	\{(6,3),(3,6),(4,5),(5,4)\}, \{(6,4),(4,6),(5,5)\}, \{(5,6),(6,5)\}, $\\$
	\{(6,6)\}\}$; \\
	$\A_Z = \{\{(6,6),(5,5),(4,4),(3,3),(2,2),(1,1)\}, $\\$
	\{(6,5),(5,6),(5,4)(4,5),(3,4)(4,3),(2,3),(3,2),(2,1),(1,2)\}, $\\$
	\{(6,4),(4,6),(5,3),(3,5),(4,2),(2,4),(3,1),(1.3)\}, $\\$
	\{(6,3),(3,6),(5,2),(2,5),(4,1),(1,4)\},$\\$
	\{(6,2),(2,6),(5,1),(1,5)\},$\\$
	\{(6,1),(1,6)\}\}$
	\item See attached
	\item If they are independent, then $P(X \cap Z) = P(X) * P(Z)$ \\ 
	$P(X \cap Z) = \frac{2}{36}$ \\
	$P(X) = \frac{6}{36}$ \\
	$P(Z) = \frac{10}{36}$ \\
	$P(A)*P(B) = \frac{5}{108}$ \\
	Because these are not equal, the events are not independent
\end{enumerate}

\item (6 points total) Consider the experiment of randomly shuffling
$n$ students amongst $n$ desks, as we did in studying de Montmort's problem.
Let $A$ be the event that ``neither student 1 nor student 2 gets
put back in their own desk".
Let $X$ be the indicator function for $A$ and let $Y = $ the number of
students do \emph{not} get put back in their own desk.

\begin{itemize}
\item (4 points) Define a sample space for the experiment and find 
$\Range(X), \A_X, f_X$.
\item (2 points) Are $X$ and $Y$ independent?  Explain.
\end{itemize}

\begin{itemize}
\item The sample space contains every possible way to arrange the n students in 
the n desks: \{(1, 2, \dots n), (2, 1, 3 \dots n) \dots\dots (n, n-1, \dots 1)\}
It as a cardinality of n!.
\item $\Range(X) = [0, 1]$, $\A_X$ 
\item They should not be independent because the number of students not that 
their original desks will influence the probability that students 1 and 2 are 
not at their original desks. For example, if only one student is not at his 
original desk, then the probability that student 1 and 2 are not at their desks 
is 0, showing that the outcome of X affects that of Y.
\end{itemize}

\item \textbf{Extra credit (4 points)} A true-false question has been
posed to a husband-and-wife team on a quiz show.  Each of the husband
and wife will independently give the correct answer with probability $p>0$
(and be wrong the rest of the time).
Which of the following is a better strategy for answering the question 
correctly?  (Explain your answer.)
\begin{itemize}
\item Let one person (say the wife) answer the question, and ignore the other.
\item Have the husband and wife consult on their answers, and if they agree
then use their answer.  Otherwise guess true of false at random.
\end{itemize}
\end{enumerate}

\clearpage
5c. 

$f_X$ 
	\begin{tabular}{|r | l|}
	\hline
	Partition & Probability \\ \hline
	2 & $\frac{1}{36}$ \\ \hline
	3 & $\frac{2}{36}$ \\ \hline
	4 & $\frac{3}{36}$ \\ \hline
	5 & $\frac{4}{36}$ \\ \hline
	6 & $\frac{5}{36}$ \\ \hline
	7 & $\frac{6}{36}$ \\ \hline
	8 & $\frac{5}{36}$ \\ \hline
	9 & $\frac{4}{36}$ \\ \hline
	10 & $\frac{3}{36}$ \\ \hline
	11 & $\frac{2}{36}$ \\ \hline
	12 & $\frac{1}{36}$ \\ \hline
	\end{tabular}

$f_Y$
	\begin{tabular}{|r | l|}
	\hline
	Partition & Probability \\ \hline
	1 & $\frac{1}{36}$ \\ \hline
	2 & $\frac{3}{36}$ \\ \hline
	3 & $\frac{5}{36}$ \\ \hline
	4 & $\frac{7}{36}$ \\ \hline
	5 & $\frac{9}{36}$ \\ \hline
	6 & $\frac{11}{36}$ \\ \hline
	\end{tabular}

$f_Z$
	\begin{tabular}{|r | l|}
	\hline
	Partition & Probability \\ \hline
	0 & $\frac{6}{36}$ \\ \hline
	1 & $\frac{10}{36}$ \\ \hline
	2 & $\frac{8}{36}$ \\ \hline
	3 & $\frac{6}{36}$ \\ \hline
	4 & $\frac{4}{36}$ \\ \hline
	5 & $\frac{2}{36}$ \\ \hline
	\end{tabular}

$f_W$



\end{document}

