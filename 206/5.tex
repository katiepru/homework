\documentclass[11pt]{article}
\usepackage{fullpage}
\usepackage{amsthm}
\usepackage{amsthm,amsmath,amsfonts,amssymb,amstext}
\usepackage{latexsym,ifthen,url,rotating}
\usepackage[usenames,dvipsnames]{color}


% --- -----------------------------------------------------------------
% --- Document-specific definitions.
% --- -----------------------------------------------------------------
\newtheorem{definition}{Definition}

\newcommand{\concat}{{\,\|\,}}
\newcommand{\bits}{\{0,1\}}
\newcommand{\Range}{{\mathrm{Range}}}
\newcommand{\A}{{\mathcal{A}}}

% --- -----------------------------------------------------------------
% --- The document starts here.
% --- -----------------------------------------------------------------
\begin{document}
%\maketitle
\sloppy

\noindent Kaitlin Poskaitis \\
Section 1\\
Homework 5\\

\begin{center}
\LARGE{\textbf{Homework 5}}\\
\large{\textbf{\emph{Due at the beginning of class on Wednesday, March 27}}}
\end{center}

\vspace{.1in}

\noindent\textbf{Instructions:} Point values for each problem are listed.
Write your solutions neatly or type them up.  Typed solutions will also be
accepted via Sakai.

\begin{enumerate}

% Q1
\item (10 points total) Two teams $x$ and $y$ are playing each other in the
World Series, which is a best-of-seven-game match that ends when one
team wins 4 games.  Assume that team $x$ wins each game with probability
$p$, and that the outcome of each game constitutes an independent trial.

\begin{enumerate}
% Q1a
\item (0.5 points) What is the probability that $x$ wins the first four games?

\begin{itemize}
	\item $p^{4}$
\end{itemize}

%Q1b
\item (2 points) What is the probability that $x$ wins four games after
at most five game have been played?
\begin{itemize}
	\item $x$ needs to win the first 4 game or four out of five games
	\item Let $A$ be the event that $x$ wins the first four games.
	\item $P(A) = p^{4}$
	\item Let $B$ be the event that $x$ plays five games and wins four
	\item $P(B) = 4(1-p)p^{4}$ (4 different ways for $x$ to win 4 out of 5 
	games)
	\item $P(A) + P(B) = p^{4}(1 + 4(1-p))$
\end{itemize}

% Q1c
\item (2 points) What is the probability that $x$ will win four games before
$y$ wins four games?  (i.e., What is the probability that $x$ wins
the Series?)

\begin{itemize}
	\item In order to find this, we need to add the probabilites that $x$ will
	win in 4, 5, 6, and 7 games.
	\item Let $A$ be the event that $x$ wins the first 4 games.
	\item $P(A) = p^{4}$
	\item Let $B$ be the event that $x$ wins in 5 games.
	\item $P(B) = 4(1-p)p^{4}$
	\item Let $C$ be the event that $x$ wins in 6 games, meaning $x$ wins 3 of 
	the first 5 games, and the 6th game.
	\item $P(C) = p(\frac{5!}{3!2!}*p^{3}(1-p)^2)$
	\item Let $D$ be the event that $x$ wins in 7 games, meaning $x$ wins 5 of
	the first 6 games, and the 7th game.
	\item $P(D) = p(\frac{6!}{3!3!}*p^{3}(1-p)^{3})$
	\item This sum can be written as a summation: 
	$\displaystyle\sum\limits_{i=4}^7 \frac{(i-1)!}{3!(i-4)!}p^{4}(1-p)^{i-4}$
\end{itemize}

%Q1d
\item (0.5 points) Calculate and simplify your answer in part (c) when $p=1/2$
and when $p=2/3$.

\begin{itemize}
	\item The probability that $x$ will win the series when $p=1/2$ is 
	$\frac{1}{2}$
	\item The probability that $x$ will win the series when $p=2/3$ is
	$\frac{1808}{2187}$
\end{itemize}

%Q1d
\item (1 point) Let $X$ be the random variable that counts the number of games
that are played.  What is $\Range(X)$?

\begin{itemize}
	\item $[4, 7]$
\end{itemize}

%Q1e
\item (2 points) What is $P(X=7)$?

\begin{itemize}
	\item In order for this to be true, $x$ must win 3 out of the first 6 games,
	and lose 3 out of the first 6 games. The outcome of the 7th game does not 
	matter.
	\item Number of ways $x$ can win 3 out of 6: $\frac{6!}{3!3!}$
	\item Therefore, the probability that $x$ wins 3 out of the first 6 games is
	$\frac{6!}{3!3!} * p^{3}(1-p)^{3}$
	\item Because the outcome of the last game does not matter,$P(X=7) = 
	2(\frac{6!}{3!3!} * p^{3}(1-p)^{3})$
\end{itemize}

%Q1f
\item (2 points) What is $P(X\geq 6)$?

\begin{itemize}
	\item In order for this to occur, $X=7$ or $X=6$
	\item Let $A$ be the event that $X=7$
	\item $P(A) = 2(\frac{6!}{3!3!} * p^{3}(1-p)^{3})$
	\item Let $B$ be the event that $X=6$
	\item In order for $B$ to occur, $x$ must win 3 of the first 5 games and the
	last game, or $x$ must lose 3 of the first 5 games and the last game.
	\item The number of ways a team could win 3 of the fist 5 games is 
	$\frac{5!}{3!2!}$
	\item Let $C$ be the event that $x$ wins 3 out of the first 5 games and the
	last game
	\item $P(C) = p(\frac{5!}{3!2!}*p^{3}(1-p)^{2})$
	\item Let $D$ be the event that $x$ loses 3 out of the first 5 games and the
	last game.
	\item $P(D) = (1-p)(\frac{5!}{3!2!}*p^{2}(1-p)^{3})$
	\item Therefore, $P(X\geq 6) = (1-p)(\frac{5!}{3!2!}*p^{2}(1-p)^{3}) + 
	p(\frac{5!}{3!2!}*p^{3}(1-p)^{2}) + 2(\frac{6!}{3!3!} * p^{3}(1-p)^{3})$
\end{itemize}
\end{enumerate}

%Q2
\item (4 points) Suppose we roll two fair dice.  Let the random variable $X=$
``the minimum of the two dice'' and $Y =$ ``the absolute value of the
difference of the two dice''.  Find $E(X)$ and $E(Y)$.

\begin{itemize}
	\item $E(X)$
	\begin{itemize}
		\item Range = $[1, 6]$
		\item Total combinations of dice rolls: 36
		\item $P(X=1) = \frac{11}{36}$
		\item $P(X=2) = \frac{9}{36}$
		\item $P(X=3) = \frac{7}{36}$
		\item $P(X=4) = \frac{5}{36}$
		\item $P(X=5) = \frac{3}{36}$
		\item $P(X=6) = \frac{1}{36}$
		\item $E(X) = \displaystyle\sum\limits_{i=1}^6 i*P(X=i) = \frac{91}{36}$
	\end{itemize}
	\item $E(Y)$
	\begin{itemize}
		\item Range = $[0, 5]$
		\item $P(X=0) = \frac{6}{36}$
		\item $P(X=1) = \frac{10}{36}$
		\item $P(X=2) = \frac{8}{36}$
		\item $P(X=3) = \frac{6}{36}$
		\item $P(X=4) = \frac{4}{36}$
		\item $P(X=5) = \frac{2}{36}$
		\item $E(Y) = \displaystyle\sum\limits_{i=0}^5 i*P(Y=i) = \frac{70}{36}$
	\end{itemize}
\end{itemize}

%Q3
\item (4 points) Suppose boxes of cereal are filled with a random prize,
each drawn from independently and uniformly from $6$ possible prizes.
If we buy $N$ boxes of cereal, what is the expected number of distinct
prizes we will collect? \begin{small}\textsf{Hint: Consider the indicator
random variables $I_{E_i}$ for the event $E_i =$ ``the $i$-th price was
in some box''.}\end{small}

\begin{itemize}
	\item Let $R_{1}...R_{6}$ be the indication random variables which are equal
	to 1 if the prize has been obtained at least once and 0 otherwise.
	\item For each box, we have a $\frac{1}{6}$ chance of getting a particular
	prize. This can also be written as $1 - \frac{5}{6}$
	\item As the number of boxes increases, the probability of getting a 
	specific toy at least once increases like so: $1 - (\frac{5}{6})^{N}$
	\item Because we have 6 indicator variables, we mulitply this number by
	6 to get the expected number of unique prizes: $6(1-(\frac{5}{6})^{N})$
\end{itemize}

%Q4
\item (4 points) A group of $m$ men and $w$ randomly sit in a single row at a
theater.  If a man and woman are seated next to each other we say they form a
couple.  (Couples can overlap, meaning that one person can be a member of two
couples.)  What is the expected number of couples?
\begin{small}\textsf{Hint: Use indicator
random variables for each possible couple forming.
}\end{small}

\begin{itemize}
	\item We will induct on a specific woman. If she is sitting on the end of a 
	row, there are $m(m+w-1)!$ permutations of the other people where she is
	part of a couple.
	\item Next, if she is in the middle of the row, she can be part of 1 or 2 
	couples. There are $2m(w-1)(m+w-3)!$ permutations of the other people where 
	she will be part of one couple. There are another $m(m-1)(m+w-3)!$ 
	permutations where she is part of 2 couples.
	\item Since we are inducting on a specific woman, we need to take into 
	account her possible permutations, and there are $(m+w-1)!$ of those.
	\item We can express this as the following:
	$\frac{2}{m+w} * \frac{m(m+w-1)!}{(m+w-1)!} + \frac{m+w-2}{m+w} * 
	\frac{2m(m+w-3)!(m+w-2)}{(m+w-1)!}$
	\item This is equal to $\frac{2m(m+w-1)}{(m+w)(m+w-1)} = \frac{2m}{m+w}$
	\item Because we inducted on one woman, we need to multiply this by $w$
	to account for all women, resulting in $\frac{2mw}{m+w}$
\end{itemize}

%Q5
\item (3 points) Suppose an experiment tosses a fair coin twice;  the experiment
``succeeds'' if both tosses were Heads.  We repeat this experiment 
for 12 independent trials.  Let $N$ be the random variable that counts
the fraction of trials that are successful (so $N = S/12$, where
$S$ is the number of successful trials).  Find $E(N)$.

\begin{itemize}
	\item The probability that a trial is successful is $1/4$
	\item Therefore, the $E(S) = \frac{12}{4} = 3$
	\item Therefore, $E(N) = \frac{3}{12} = \frac{1}{4}$
\end{itemize}

%Q6
\item \textbf{Extra Credit: (4 points)} Consider the experiment where $n$ balls
are to be placed randomly into $n$ boxes. Let $N_1$ count the number of boxes
with exactly one ball, and let $N_2$ count the number of boxes with exactly two
balls. Find the probability of the events ``$N_1 = n$" and ``$N_1 = n - 1$".
Use the indicator technique to find $E(N_1)$ and $E(N_2)$.

\begin{itemize}
	\item Probability $N_1 = n$: $\frac{1}{{2n-1\choose n-1}}$
	\item Probability $N_1 = n-1$ is $0$
\end{itemize}


\end{enumerate}


\end{document}
